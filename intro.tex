Aux prémisses de la club culture le rôle des DJs était essentiellement de jouer des enregistrements musicaux pour un public, en général dansant. Avec l’évolution rapide de la technologie et en particulier l’omniprésence de l’audionumérique à partir des années 2000, de nombreux DJs sont devenus à la fois compositeurs, producteurs et performeurs de leurs propres morceaux. C'est notamment le cas des artistes évoluant dans le milieu de l'Electronic Dance Music (abbréviée EDM)\footnote{\emph{Electronic Dance Music} (EDM) est un terme parapluie regroupant de nombreux genres de musique électronique tels que la house, la techno, la trance, la drum n bass, le dubstep, etc. }.

Ces artistes composent le plus souvent leur musique à l’aide d’une station audionumérique (en anglais Digital Audio Workstation ou DAW) qui leur permet de combiner sampleurs, synthétiseurs et enregistrements pour créer un morceau de musique complet. Recréer en live un tel morceau, composé souvent de plusieurs dizaines de pistes, reste difficile et conditionné par les contrôles proposés par les DAW. Lors de leur performance live, les musiciens et musiciennes d’EDM utilisent donc des techniques de DJing leur permettant de mixer des pistes créées au préalable en appliquant des effets sur celles-ci. En revanche, ils n’ont pas ou peu accès à la structure interne du morceau qu’ils produisent. Modifier par exemple la structure rythmique ou harmonique d’un morceau joué en live est difficilement réalisable dans ce contexte. 

Cet article présente LiveScaler\footnote{LiveScaler et son code source sont disponibles en libre accès sur Github : \href{https://github.com/autonym8/LiveScaler}{github.com/autonym8/LiveScaler}}, qui permet de modifier en live la structure harmonique d'un morceau de musique électronique. En partant d'un morceau composé au préalable, LiveScaler permet d'appliquer des transformations MIDI à l'ensemble de ses instruments virtuels. LiveScaler conserve l'ensemble des caractéristiques du morceau initial mais agit sur la hauteur des notes, permettant de changer en live l'harmonie du morceau ou de générer de nouvelles mélodies. En particulier, l'accent est mis sur la possibilité d'utiliser LiveScaler dans un DAW, ici Ableton Live, et en minimisant les contraintes et les connaissances techniques nécessaires pour son utilisation.

Cet article est organisé de la manière suivante : dans un premier temps, nous définirons les transformations affines, un ensemble restreint de transformations de l'espace des hauteurs de notes. Ensuite, nous présenterons Live\-Scaler, qui implémente l'application de ces transformations en live à un nombre arbitraire d'instruments virtuels. Enfin, nous décrirons la manière dont nous avons utilisé LiveScaler dans le cadre d'une performance\footnote{Une vidéo de démonstration se trouve à l'adresse suivante : \href{https://youtu.be/qxdbOffQPvU}{youtu.be/qxdbOffQPvU}} live d'EDM . Nous terminerons par une comparaison de LiveScaler avec des travaux et outils préexistants.