Aux origines de la \emph{club culture}, le rôle des DJs était essentiellement de jouer des enregistrements musicaux pour un public, en général dansant. Avec l’évolution rapide de la technologie et en particulier l’omniprésence de l’audionumérique à partir des années 2000, de nombreux DJs sont devenus à la fois compositeurs, producteurs et performeurs de leurs propres morceaux. C'est notamment le cas des artistes évoluant dans le milieu de l'Electronic Dance Music (EDM)\footnote{\emph{Electronic Dance Music} (EDM) est un terme parapluie regroupant de nombreux genres de musique électronique tels que la house, la techno, la trance, la drum n bass, le dubstep, etc. }.


Ces artistes composent le plus souvent leur musique à l’aide d’une station audionumérique (Digital Audio Workstation ou DAW) qui leur permet de combiner sampleurs, synthétiseurs et enregistrements pour créer un morceau de musique complet. Recréer en live un tel morceau, composé souvent de plusieurs dizaines de pistes, reste difficile et conditionné par les contrôles proposés par les DAW. Pour leur performance live, la plupart des artistes vont ainsi choisir entre interpréter certaines pistes spécifiques à l'aide de synthétiseurs ou sampleurs, remixer en live un morceau préparé au préalable ou utiliser des techniques de DJing leur permettant de mixer entre eux des rendus audios en leur appliquant divers effets \cite{ferreira2008sound},\cite{magana2018performance}.  Dans ces contextes,  modifier le rythme ou l'harmonie d’un morceau joué en live est difficilement réalisable dans ces contextes. En effet, cela requerrait un accès à la structure interne du morceau, ce que le DJing ne permet pas, ou bien de contrôler toutes les pistes simultanément, ce qui est précisément ce que nous proposons de faire ici.

Cet article présente LiveScaler\footnote{LiveScaler et son code source sont disponibles en libre accès sur Github : \href{https://github.com/autonym8/LiveScaler}{github.com/autonym8/LiveScaler}}, qui per\-met de modifier en live la structure harmonique d'un morceau de musique électronique. En partant d'un morceau composé au préalable, LiveScaler permet d'appliquer des transformations MIDI à l'ensemble des instruments virtuels\footnote{Une vidéo de démonstration est disponible à l'adresse suivante : \href{ https://youtu.be/Cn0HBgWS5Pw}{youtu.be/Cn0HBgWS5Pw}}. LiveScaler conserve l'ensemble des caractéristiques du morceau initial mais agit sur la hauteur des notes, permettant de changer en live l'harmonie du morceau ou de générer de nouvelles mélodies. En particulier, l'accent est mis sur la possibilité d'utiliser LiveScaler dans un DAW, ici Ableton Live, tout en minimisant les contraintes et les connaissances techniques nécessaires pour son utilisation.

Cet article est organisé de la manière suivante : nous commencerons par définir les transformations affines, un ensemble restreint de transformations de l'espace des hauteurs de notes. Ensuite, nous présenterons Live\-Scaler, qui implémente l'application de ces transformations en live à un nombre arbitraire d'instruments virtuels. Enfin, nous décrirons la manière dont nous avons utilisé LiveScaler dans le cadre d'une performance live d'EDM . Nous terminerons par une comparaison de LiveScaler avec les travaux et outils existants.