\documentclass{article}
\usepackage{jim,amsmath}
\usepackage[utf8]{inputenc}
\usepackage[french]{babel}
\usepackage[T1]{fontenc}

\usepackage[table]{xcolor}

\usepackage{tikz}
\usetikzlibrary{arrows,decorations.markings}
\usetikzlibrary{cd}
\usetikzlibrary{shapes.geometric,fit}
\usetikzlibrary{positioning}
\tikzcdset{every label/.append style = {font = \tiny}}

\usepackage{leadsheets}

\usepackage{minted}
\setkeys{Gin}{width=\linewidth,totalheight=\textheight,keepaspectratio}

\usepackage{mathtools}
\usepackage{amsmath}
\usepackage{amsfonts}
\usepackage{amsthm}
\usepackage{amssymb}

\usepackage{graphicx}
\graphicspath{{Figures/}}
\usepackage{wrapfig}
\usepackage{float}

\usepackage{hyperref}
\interfootnotelinepenalty=10000

\usepackage{setspace}
\onehalfspacing

%\usepackage{caption}
\usepackage{subcaption}


\usepackage{csquotes}
\usepackage{comment}


\usepackage{tabularx}
\definecolor{tableShade}{gray}{0.9}

\usepackage[shortlabels]{enumitem}



\newcommand{\LSHist}{\texttt{Hist} }
\newcommand{\LSMod}{\texttt{Mod} }
\newcommand{\LSpp}{\texttt{++} }
\newcommand{\LSmm}{\texttt{-\hspace{0.5pt}-} }
\newcommand{\LSI}{\writechord{I}}
\newcommand{\LSII}{\writechord{II}}
\newcommand{\LSIII}{\writechord{III}}
\newcommand{\LSIV}{\writechord{IV}}
\newcommand{\LSV}{\writechord{V}}
\newcommand{\LSVI}{\writechord{VI}}
\newcommand{\LSVII}{\writechord{VII}}
\newcommand{\LSi}{\writechord{i}}
\newcommand{\LSii}{\writechord{ii}}
\newcommand{\LSiii}{\writechord{iii}}
\newcommand{\LSiv}{\writechord{iv}}
\newcommand{\LSv}{\writechord{v}}
\newcommand{\LSvi}{\writechord{vi}}
\newcommand{\LSvii}{\writechord{vii}}
\newcommand{\LStwo}{\texttt{2}}
\newcommand{\LSthree}{\texttt{3}}
\newcommand{\LSfour}{\texttt{4}}
\newcommand{\LSMm}{\texttt{M}$\leftrightarrow$\texttt{m} }

\DeclareRobustCommand{\svdots}{% s for `scaling'
  \vcenter{%
    \offinterlineskip
    \hbox{.}
    \vskip0.25\normalbaselineskip
    \hbox{.}
    \vskip0.25\normalbaselineskip
    \hbox{.}%
  }%
}


\title{LiveScaler : Contrôler en live l'harmonie d'un morceau de musique électronique}
\oneauthor
 {Alice Rixte} {Université de Bordeaux \\ alice.rixte@u-bordeaux.fr }

\begin{document}  

\maketitle

%\tableofcontents
\begin{abstract}
  Lorem ipsum dolor sit amet, consectetuer adipiscing elit. Aenean commodo ligula eget dolor. Aenean massa. Cum sociis natoque penatibus et magnis dis parturient montes, nascetur ridiculus mus. Donec quam felis, ultricies nec, pellentesque eu, pretium quis, sem. Nulla consequat massa quis enim. Donec pede justo, fringilla vel, aliquet nec, vulputate eget, arcu. In enim justo, rhoncus ut, imperdiet a, venenatis vitae, justo. Nullam dictum felis eu pede mollis pretium. Integer tincidunt. Cras dapibus. Vivamus elementum semper nisi. Aenean vulputate eleifend tellus. Aenean leo ligula, porttitor eu, consequat vitae, eleifend ac, enim. Aliquam lorem ante, dapibus in, viverra quis, feugiat a, tellus. Phasellus viverra nulla ut metus varius laoreet. Quisque rutrum. Aenean imperdiet. Etiam ultricies nisi vel augue. Curabitur ullamcorper ultricies nisi. Nam eget dui. Etiam rhoncus. Maecenas tempus, tellus eget condimentum rhoncus, sem quam semper libero, sit amet adipiscing sem neque sed ipsum. Nam quam nunc, blandit vel, luctus pulvinar, hendrerit id, lorem. Maecenas nec odio et ante tincidunt tempus. Donec vitae sapien ut libero venenatis faucibus. Nullam quis ante. Etiam sit amet orci eget 
\end{abstract}

\begin{comment}
  a comme image à la place d'envoie
cohérence dans la notation des notes 
repliement de l'intervalle
ça marche pas bien quand on sort des modes diatoniques
mode de messiaen : modes à transposition limitée

Expliquiqué dans Techniques de mon langage musical ou dans son cours de composition, tome 1

expliquer dans l'introduction que ce que je veux faire, c'est du live, et faire de la musique

quelques canons stochastiques

Joe Zawinul Modules SEM ou Oberheim claviers en miroirs
\end{comment}

\section{Introduction}
Aux origines de la \emph{club culture}, le rôle des DJs était essentiellement de jouer des enregistrements musicaux pour un public, en général dansant. Avec l’évolution rapide de la technologie et en particulier l’omniprésence de l’audionumérique à partir des années 2000, de nombreux DJs sont devenus à la fois compositeurs, producteurs et performeurs de leurs propres morceaux. C'est notamment le cas des artistes évoluant dans le milieu de l'Electronic Dance Music (EDM)\footnote{\emph{Electronic Dance Music} (EDM) est un terme parapluie regroupant de nombreux genres de musique électronique tels que la house, la techno, la trance, la drum n bass, le dubstep, etc. }.


Ces artistes composent le plus souvent leur musique à l’aide d’une station audionumérique (Digital Audio Workstation ou DAW) qui leur permet de combiner sampleurs, synthétiseurs et enregistrements pour créer un morceau de musique complet. Recréer en live un tel morceau, composé souvent de plusieurs dizaines de pistes, reste difficile et conditionné par les contrôles proposés par les DAW. Lors de leur performance live, les musiciens de l'EDM utilisent donc des techniques de DJing leur per\-mettant de mixer entre eux des rendus audios de leurs propres morceaux en leur appliquant des effets audio. Par conséquent, ils n’ont pas ou peu accès à la structure interne de ces morceaux. Ainsi, modifier le rythme ou l'harmonie d’un morceau joué en live est difficilement réalisable dans ce contexte.

Cet article présente LiveScaler\footnote{LiveScaler et son code source sont disponibles en libre accès sur Github : \href{https://github.com/autonym8/LiveScaler}{github.com/autonym8/LiveScaler}}, qui per\-met de modifier en live la structure harmonique d'un morceau de musique électronique. En partant d'un morceau composé au préalable, Live\-Scaler permet d'appliquer des transformations MIDI à l'ensemble des instruments virtuels. LiveScaler conserve l'ensemble des caractéristiques du morceau initial mais agit sur la hauteur des notes, permettant de changer en live l'harmonie du morceau ou de générer de nouvelles mélodies. En particulier, l'accent est mis sur la possibilité d'utiliser LiveScaler dans un DAW, ici Ableton Live, tout en minimisant les contraintes et les connaissances techniques nécessaires pour son utilisation.

Cet article est organisé de la manière suivante : nous commencerons par définir les transformations affines, un ensemble restreint de transformations de l'espace des hauteurs de notes. Ensuite, nous présenterons Live\-Scaler, qui implémente l'application de ces transformations en live à un nombre arbitraire d'instruments virtuels. Enfin, nous décrirons la manière dont nous avons utilisé LiveScaler dans le cadre d'une performance\footnote{Une vidéo de démonstration est disponible à l'adresse suivante : \href{ https://youtu.be/Cn0HBgWS5Pw}{youtu.be/Cn0HBgWS5Pw}} live d'EDM . Nous terminerons par une comparaison de LiveScaler avec les travaux et outils existants.
\section{Transformations de gammes}
\begin{comment}
  \subsection{Quelles transformations de gamme autoriser ?}
Les transformation affines de la forme $an + b$ où $n$ est la note de départ.
\subsection{Pourquoi ces transformations ?}
\begin{enumerate}
  \item Ces transformations sont adaptées à la musique tonale occidentale : passage du majeur au mineur.
  \item Elles sont facilement implémentables
  \item Elles peuvent être exprimées par une paire d'entiers, ce qui permet de les communiquer directement via MIDI.
\end{enumerate}


\subsection{Peut-on quand même sortir de la tonalité ?}
Oui : \begin{enumerate}
  \item on peut passer d'une gamme quelconque à une gamme par tons
  \item elles s'appliquent dans un contexte microtonal
  \item Possibilité d'ajouter une permutation quelconque personnalisée
\end{enumerate} 
\end{comment}
Dans cet article, on se place dans le contexte des tempéraments à division multiple. On notera $b$ le nombre de divisions de l'octave. Bien que l'ensemble des exemples proposés ici se focalisent sur le tempérament égal $b = 12$, le lecteur se persuadera aisément que tout ce qui est proposé ici peut se généraliser à tous les tempéraments à division multiple. On peut alors associer à toute note un entier $n\in \mathbb{Z}$. En particulier, nous utiliserons ici la norme MIDI en l'étendant à l'ensembles des entiers relatifs : $B_{-2}$ correspond ainsi à $-1$, $C_{-1}$ à $0$, $C\sharp_{-1}$  à $1$, $A_3$ à $57$ \dots

Nous définissons ici une \emph{transformation de gamme} comme une fonction qui à toute hauteur de note associe une nouvelle hauteur de note a priori quelconque, autrement dit une fonction $\mathbb{Z}$ dans $\mathbb{Z}$.

Un bon exemple de transformation de gamme est la transposition : à chaque note $n$ on associe la note $n+\tau$  décalée de $\tau$ demi-tons vers l'aigu lorsque $\tau$ est positif et vers le grave lorsque $\tau$ est négatif. Ainsi, une transposition de $\tau$ demi-tons est représentée par la transformation de gamme $ n \mapsto n+\tau$ (voir Figure \ref{fig:transp}).


\subsection{Transformations affines}

Les transformations de gamme qui vont nous intéresser ici sont les \emph{transformation affines}\footnote{Cette définition est inspirée par les automorphismes du groupe $T/I$ présentés par \textcite{lewin1990klumpenhouwer}.}, c'est-à-dire les fonctions de la forme $A\langle\mu,\tau\rangle : n \mapsto \mu n + \tau$ avec $\mu$ le \emph{coefficient modal} de la transformation affine et $\tau$ le \emph{coefficient de transposition}. 

Les transformations affines ont la propriété importante de préserver les classes de hauteur \footnote{En effet, pour toute base $b\in \mathbb{N}^*$, $\forall n_1,n_2 \in \mathbb{Z}, n_1 \equiv n_2 \mod b \implies \mu n_1 + \tau \equiv \mu n_2 + \tau \mod b$. Avec $b=12$, on obtient le résultat pour les classes de hauteurs dodécaphoniques. } c'est-à-dire que si deux notes sont identiques à l'octave près, alors elles le seront toujours une fois la transformation affine appliquée. 

Nous allons à présent étudier plusieurs exemples afin de donner au lecteur un aperçu de leur expressivité.
\subsubsection{Transpositions}
\begin{figure}[htbp]
  \centering
  \begin{tikzpicture}[baseline= (a).base]    

    \node[scale=1] (a) at (0,0){
    \begin{tikzcd}[column sep=0pt, minimum width=11.5mm, row sep=0.1cm]
    {...} & {A_{-1}} & {A\sharp_{-1}} & {B_{-1}} & {C_{0}} & {C\sharp_{0}} & {D_0} & {D\sharp_0} & {E_0} & {F_0} & {F\sharp_0} & {G_0} & {...} \\
    {...} & {-3} & {-2} & {-1} & 0 & 1 & 2 & 3 & 4 & 5 & 6 & 7 & {...} \\
    {} &&&&&&&&&&& {} \\
    {} &&&&&&&&&&& {} \\
    {} &&&&&&&&&&& {} \\
    {} &&&&&&&&&&& {} \\
    {} &&&&&&&&&&& {} \\
    {} &&&&&&&&&&& {} \\
    {...} & {-3} & {-2} & {-1} & 0 & 1 & 2 & 3 & 4 & 5 & 6 & 7 & {...} \\
    {...} & {A_{-1}} & {A\sharp_{-1}} & {B_{-1}} & {C_0} & {C\sharp_0} & {D_0} & {D\sharp_0} & {E_0} & {F_0} & {F\sharp_0} & {G_0} & {...}
    \arrow[color={rgb,255:red,117;green,117;blue,117}, dotted, from=2-1, to=9-3]
    \arrow[from=2-2, to=9-4]
    \arrow[from=2-3, to=9-5]
    \arrow[from=2-4, to=9-6]
    \arrow[from=2-5, to=9-7]
    \arrow[from=2-6, to=9-8]
    \arrow[from=2-7, to=9-9]
    \arrow[from=2-8, to=9-10]
    \arrow[from=2-9, to=9-11]
    \arrow[from=2-10, to=9-12]
    \arrow[color={rgb,255:red,117;green,117;blue,117}, dotted, from=2-11, to=9-13]
    \end{tikzcd}
  };
  \end{tikzpicture}
  \caption{La transformation $A \langle 1,2 \rangle : n \mapsto n + 2$ correspond à la  transposition d'un ton vers l'aigu}\label{fig:transp}
\end{figure}
Lorsque $\mu = 1$, les transformations affines $A\langle 1,\tau \rangle : n \mapsto n + \tau$ permettent de représenter toutes les transpositions possibles (voir Figure \ref{fig:transp}).


\subsubsection{Inversions}
\begin{figure}[ht]
  \centering
  \begin{tikzpicture}[baseline= (a).base]    

    \node[scale=1] (a) at (0,0){
    \begin{tikzcd}[column sep=0pt, minimum width=11.5mm, row sep=0.1cm]
    % https://q.uiver.app/?q=WzAsNTQsWzUsMSwiQ197MH0iXSxbNiwxLCJDXFwjX3swfSJdLFs3LDEsIkRfMCJdLFs4LDEsIkRcXCNfMCJdLFs5LDEsIkVfMCJdLFsxMCwxLCJGXzAiXSxbMTEsMSwiRlxcI18wIl0sWzEyLDEsIkdfMCJdLFsxMywxLCIuLi4iXSxbOCwwXSxbNSwyLCIwIl0sWzYsMiwiMSJdLFs3LDIsIjIiXSxbOCwyLCIzIl0sWzksMiwiNCJdLFsxMCwyLCI1Il0sWzExLDIsIjYiXSxbMTIsMiwiNyJdLFs3LDNdLFs0LDEsIkJfey0xfSJdLFszLDEsIkFcXCNfey0xfSJdLFsyLDEsIkFfey0xfSJdLFsyLDIsIi0zIl0sWzMsMiwiLTIiXSxbNCwyLCItMSJdLFsxLDIsIi4uLiJdLFs5LDQsIjQiXSxbMiw0LCItMyJdLFs3LDQsIjIiXSxbNSw0LCIwIl0sWzEwLDQsIjUiXSxbMTIsNCwiNyJdLFs1LDUsIkNfMCJdLFs0LDQsIi0xIl0sWzMsNCwiLTIiXSxbNiw0LCIxIl0sWzgsNCwiMyJdLFsxMSw0LCI2Il0sWzIsNSwiQV97LTF9Il0sWzMsNSwiQVxcI197LTF9Il0sWzQsNSwiQl97LTF9Il0sWzEsNSwiLi4uIl0sWzEsMSwiLi4uIl0sWzEzLDIsIi4uLiJdLFsxMyw0LCIuLi4iXSxbMTMsNSwiLi4uIl0sWzYsNSwiQ1xcI18wIl0sWzcsNSwiRF8wIl0sWzgsNSwiRFxcI18wIl0sWzksNSwiRV8wIl0sWzEwLDUsIkZfMCJdLFsxMSw1LCJGXFwjXzAiXSxbMTIsNSwiR18wIl0sWzAsM10sWzEwLDI2XSxbMTIsMjgsIiIsMCx7ImNvbG91ciI6WzAsMCw0Nl0sInN0eWxlIjp7ImJvZHkiOnsibmFtZSI6ImRhc2hlZCJ9fX1dLFsxNCwyOV0sWzI0LDMwLCIiLDIseyJjb2xvdXIiOlswLDAsNDZdLCJzdHlsZSI6eyJib2R5Ijp7Im5hbWUiOiJkYXNoZWQifX19XSxbMjIsMzFdLFsxNSwzMywiIiwyLHsiY29sb3VyIjpbMCwwLDQ2XSwic3R5bGUiOnsiYm9keSI6eyJuYW1lIjoiZGFzaGVkIn19fV0sWzE3LDI3XV0=
      {...} & {A_{-1}} & {A\sharp_{-1}} & {B_{-1}} & {C_{0}} & {C\sharp_{0}} & {D_0} & {D\sharp_0} & {E_0} & {F_0} & {F\sharp_0} & {G_0} & {...} \\
      {...} & {-3} & {-2} & {-1} & 0 & 1 & 2 & 3 & 4 & 5 & 6 & 7 & {...} \\
      {} &&&&&&&&&&& {} \\
      {} &&&&&&&&&&& {} \\
      {} &&&&&&&&&&& {} \\
      {} &&&&&&&&&&& {} \\
      {} &&&&&&&&&&& {} \\
      {} &&&&&&&&&&& {} \\
      {...} & {-3} & {-2} & {-1} & 0 & 1 & 2 & 3 & 4 & 5 & 6 & 7 & {...} \\
      {...} & {A_{-1}} & {A\sharp_{-1}} & {B_{-1}} & {C_0} & {C\sharp_0} & {D_0} & {D\sharp_0} & {E_0} & {F_0} & {F\sharp_0} & {G_0} & {...}
      \arrow[from=2-5, to=9-9]
      \arrow[color={rgb,255:red,117;green,117;blue,117}, dashed, from=2-7, to=9-7]
      \arrow[from=2-9, to=9-5]
      \arrow[color={rgb,255:red,117;green,117;blue,117}, dashed, from=2-4, to=9-10]
      \arrow[from=2-2, to=9-12]
      \arrow[color={rgb,255:red,117;green,117;blue,117}, dashed, from=2-10, to=9-4]
      \arrow[from=2-12, to=9-2]
    \end{tikzcd}
    };
  \end{tikzpicture}   
  \caption{La transformation $A_{-1,4} :n\mapsto -n + 4$ correspond au passage à la relative mineure de $C$ à $Am$\label{fig:inversion}}
  \medskip
  \small
  Pour des raisons de lisibilité seules les flèches de la gamme de do majeur ont été tracées. On notera la passage de l'accord $C$ à $Am$ et de $Am$ à $C$.
\end{figure}

Lorsque $\mu = -1$, les transformations $A \langle -1,\tau\rangle : n\mapsto -n + \tau$ permettent de passer d'une gamme majeure à une gamme mineure et réciproquement. La Figure \ref{fig:inversion} illustre la manière dont le transformation $A\langle -1,4 \rangle$ envoie l'accord de Do majeur $C_0,E_0,G_0$ vers un accord de La mineur $A_{-1},C_0,E_0$ mais aussi envoie l'accord de la mineur $A_{-1},C_0,E_0$ sur l'accord de do majeur  $C_0,E_0,G_0$. Comme les transformations affines préservent les classes de hauteur, on peut considérer que $A \langle -1,4\rangle$ envoie Do majeur sur La mineur et La mineur sur Do majeur. Le tableau \ref{tab:triadesA-14} explicite l'image des accords de la gamme de Do majeur par $A\langle -1,4 \rangle$.



\begin{table}[ht]
  
  \centering % instead of \begin{center}
  \begin{tabular}{ccc}
      \writechord{Cma} & $\mapsto$ & \writechord{Ami}\\
      \writechord{Dmi} & $\mapsto$ & \writechord{Gma}\\
      \writechord{Emi} & $\mapsto$ & \writechord{Fma}\\
      \writechord{Fma} & $\mapsto$ & \writechord{Emi}\\
      \writechord{Gma} & $\mapsto$ & \writechord{Dmi}\\
      \writechord{Ami} & $\mapsto$ & \writechord{Cma}\\
      \writechord{Bo} & $\mapsto$ & \writechord{Bo}
  \end{tabular}
  \caption{Image des accords de la gamme de do majeur par $A\langle -1, 4 \rangle$\label{tab:triadesA-14}}
\end{table}

\subsubsection{Transformation vers un mode à transposition limitée}
Lorsque $\mu = 2$ ou $\mu = -2$, les transformations affines envoient n'importe quelle gamme vers une gamme apparentée à une gamme par tons (voir Figure \ref{fig:gammepartons}). Les transformations affines peuvent donc permettre de sortir du cadre de la musique tonale occidentale.

\begin{figure}[htbp]
  \centering
  \begin{tikzpicture}[baseline= (a).base]    

    \node[scale=1] (a) at (0,0){
      \begin{tikzcd}[column sep=0mm, minimum width = 0mm, minimum height=7mm, row sep=0cm]
        \svdots   & \svdots & \hspace{20mm} & \svdots & \svdots \\
        \writechord{G}_{5}  & 7  & & 7  & \writechord{G}_{5}  \\
        \writechord{F#}_{5} & 6  & & 6  & \writechord{F#}_{5} \\
        \writechord{F}_{5}  & 5  & & 5  & \writechord{F}_{5}  \\
        \writechord{E}_{5}  & 4  & & 4  & \writechord{E}_{5}  \\
        \writechord{D#}_{5} & 3  & & 3  & \writechord{D#}_{5} \\
        \writechord{D}_{5}  & 2  & & 2  & \writechord{D}_{5}  \\
        \writechord{C#}_{5} & 1  & & 1  & \writechord{C#}_{5} \\
        \writechord{C}_{5}  & 0  & & 0  & \writechord{C}_{5}  \\
        \writechord{B}_{4}  & -1 & & -1 & \writechord{B}_{4}  \\
        \writechord{A#}_{4} & -2 & & -2 & \writechord{A#}_{4} \\
        \writechord{A}_{4}  & -3 & & -3 & \writechord{A}_{4}  \\
        \svdots         & \svdots & & \svdots &         \svdots 
        \arrow[from=5-2, to=1-4, color={rgb,255:red,117;green,117;blue,117}, dotted]
        \arrow[from=6-2, to=3-4]
        \arrow[from=7-2, to=5-4]
       \arrow[from=8-2, to=7-4]
       \arrow[from=9-2, to=9-4]
       \arrow[from=10-2, to=11-4]
       \arrow[from=11-2, to=13-4, color={rgb,255:red,117;green,117;blue,117}, dotted]
    \end{tikzcd}
    };
  \end{tikzpicture}   
  \caption{La transformation $A\langle 2,0\rangle :n\mapsto 2n$ permet d'obtenir des gammes apparentées à la gamme par tons.\label{fig:gammepartons}}
\end{figure}

Contrairement aux inversions et aux transpositions, cette transformation n'est pas bijective : l'image de $A\langle 2,0 \rangle$ contient exactement $6$ classes de hauteurs qui correspondent aux $6$ notes d'une des deux gammes par tons. Il est intéressant de noter que l'image d'une gamme majeure ou mineure naturelle par $A\langle 2,\tau\rangle$ contient les $6$ notes de la gamme par tons (voir Table \ref{tab:minparton}). Ce n'est pas le cas pour la gamme mineure harmonique dont l'image par $A\langle 2,\tau\rangle$ ne contient que $5$ classes de hauteur.




\begin{table}[htbp]
  \centering
  \begin{subtable}{0.45\columnwidth}
    \centering % instead of \begin{center}
      \begin{tabular}{ccc}
          \writechord{C} & $\mapsto$ & \writechord{C}\\
          \writechord{D} & $\mapsto$ & \writechord{E}\\
          \writechord{E} & $\mapsto$ & \writechord{G\sharp}\\
          \writechord{F} & $\mapsto$ & \writechord{A\sharp}\\
          \writechord{G} & $\mapsto$ & \writechord{D}\\
          \writechord{A} & $\mapsto$ & \writechord{F\sharp}\\
          \writechord{B} & $\mapsto$ & \writechord{A\sharp}
      \end{tabular}
  \end{subtable}
  \begin{subtable}{0.45\columnwidth}
      \centering % instead of \begin{center}
      \begin{tabular}{ccc}
          \writechord{C} & $\mapsto$ & \writechord{C}\\
          \writechord{D} & $\mapsto$ & \writechord{E}\\
          \writechord{E\flat} & $\mapsto$ & \writechord{F\sharp}\\
          \writechord{F} & $\mapsto$ & \writechord{A\sharp}\\
          \writechord{G} & $\mapsto$ & \writechord{D}\\
          \writechord{A\flat} & $\mapsto$ & \writechord{E}\\
          \writechord{B\flat} & $\mapsto$ & \writechord{G\sharp}
      \end{tabular}
    \end{subtable}
    \caption{Image des gammes de C majeur (à gauche) et C mineur naturel(à droite) par $A\langle 2, 0 \rangle$\label{tab:minparton}}
\end{table}

De manière plus générale, lorsque le coefficient modal $\mu$ n'est pas premier avec $\beta$, on obtient un mode à transposition limitée dont les notes sont séparées par un intervalle de $\mu$ demi-tons\footnote{On montre en effet que le nombre de classes de hauteur dans l'image de $A\langle \mu, \tau\rangle$ est égal à $\frac{12}{\mu\wedge 12}$ où $\wedge$ dénote le pgcd de deux entiers.}. Ainsi, pour $\mu = 4$, on obtient un mode composé de $4$ classes d'hauteurs, séparées par des tierces mineures.


La Table \ref{tab:classmu} résume les différents types de transformations qu'offrent les transformations affines, ainsi que le nombre de classes de hauteur dans l'image de ces transformations. Les transformations affines bijectives - leur image contient $12$ classes de hauteurs - correspondent  aux automorphismes $F\langle u,j \rangle$ du groupe $T/I$ décrits par \cite{lewin1990klumpenhouwer}.


\begin{table}[htbp]
  \centering
  \rowcolors{2}{gray!25}{white}
  \begin{tabular}{ccc}
    \rowcolor{gray!50}
    $\mu$ & Type de transformation & Classes de hauteur\\
    -1 & Inversions majeur/mineur & 12\\
    0 & Octaves & 1\\
    1 & Transpositions & 12 \\
    -2,2 & Gamme par tons & 6 \\
    -3,3 & Tierces mineures &4 \\
    -4,4 & Tierces majeures & 3\\
    -5,5 & $F\langle 5,\tau \rangle$, $F\langle 7,\tau \rangle$& 12 \\
    -6,6 & Tritons & 2\\
  \end{tabular}
  \caption{Classification des transformations affines en fonction de leur coefficient modal $\mu$\label{tab:classmu} } 
\end{table}
\subsubsection{Coefficient modal}

La Table \ref{tab:classmu} résume les différents types de transformations qu'offrent les transformations affines, ainsi que le nombre de classes de hauteur dans l'image de ces transformations \footnote{On montre aisément que le nombre de classes de hauteur dans l'image de $A\langle \mu, \tau\rangle$ est égal à $\frac{12}{\mu\wedge 12}$ où $\wedge$ dénote le pgcd de deux entiers.}. Les transformations affines bijectives - leur image contient $12$ classes de hauteurs - correspondent  aux automorphismes $F\langle u,j \rangle$ du groupe $T/I$ décrits par \cite{lewin1990klumpenhouwer}.


\begin{table}[h]
  \centering
  \rowcolors{2}{gray!25}{white}
  \begin{tabular}{ccc}
    \rowcolor{gray!50}
    $\mu$ & Type de transformation & Classes de hauteur\\
    -1 & Inversions majeur/mineur & 12\\
    0 & Octaves & 1\\
    1 & Transpositions & 12 \\
    -2,2 & Gamme par tons & 6 \\
    -3,3 & Tierces mineures &4 \\
    -4,4 & Tierces majeures & 3\\
    -5,5 & $F\langle 5,\tau \rangle$, $F\langle 7,\tau \rangle$& 12 \\
    -6,6 & Tritons & 2\\
  \end{tabular}
  \caption{Classification des transformations affines en fonction de leur coefficient modal $\mu$\label{tab:classmu} } 
\end{table}
\subsection{Ancres}

Les paramètres des transformations $A\langle \mu, \tau \rangle$ dépendent de la tonalité dans laquelle on se trouve. Ainsi le passage à le relative mineure est $A\langle -1,4 \rangle$ lorsque la tonalité est Do majeur mais correspond à $A\langle -1,6 \rangle$ lorsqu'on est en Sol majeur.

Pour éviter les confusions, on ajoute un troisième paramètre $\alpha$, appelé \emph{ancre} à nos transformations affines. L'ancre correspond à la note qui sert de référence pour appliquer la transformation affine considérée, c'est-à-dire à $0$ dans l'espace $\mathbb{Z}$ des hauteurs de notes. Pour une transformation quelconqe $T:\mathbb{Z} \rightarrow \mathbb{Z}$ et une ancre $\alpha$, on obtient une nouvelle transformation $T\langle \alpha \rangle = A\langle 1, \alpha \rangle \circ T \circ A\langle 1, -\alpha \rangle $ centrée en $\alpha$. 



S'il est clair que les ancres n'ajoutent aucune expressivité à nos transformations affines, elles permettent de rendre la modulation et la transposition avec LiveScaler extrêmement intuitives. En prenant pour référence une gamme majeure dont la tonique est donnée par l'ancre $\alpha$, nous pouvons nommer certaines transformations affines en fonction de l'image de l'accord de tonique. Ainsi les transpositions de $7$ demi-tons vers le haut  $A\langle 1, 7, \alpha\rangle$ envoient l'accord de tonique sur l'accord de dominante et sont donc notées \emph{V}. Le passage à la relative mineure $A\langle -1, 4, \alpha \rangle$ correspond ainsi à \emph{vi}. La Table \ref{tab:degrees} donne les transformations affines pour chaque triade d'une gamme majeure.

\begin{table}[htbp]
  \centering
  \rowcolors{2}{gray!25}{white}
  \begin{tabular}{ccc}
    \rowcolor{gray!50}
    Degré & Transformation affine\\
    \writechord{I} & $A\langle ~~1, ~~0, \alpha \rangle$\\
    \writechord{II} &  $A\langle ~~1, ~~2, \alpha \rangle$\\
    \writechord{iii} &  $A\langle -1, -1, \alpha \rangle$\\
    \writechord{IV} &  $A\langle ~~1,~~ 5, \alpha \rangle$\\
    \writechord{V} &  $A\langle ~~ 1, ~~7, \alpha \rangle$\\
    \writechord{vi}& $A\langle -1, ~~4, \alpha \rangle$\\
    \writechord{vii} & $A\langle -1, ~~6, \alpha \rangle$\\
  \end{tabular}
  \caption{ Correspondances entre triades d'une gamme majeure et transformations de gamme\label{tab:degrees} } 
\end{table}






\subsection{Restriction de l'écart de hauteur}
Jusqu'ici, nous avons présenté les transformations affines en nous concentrant sur leur action sur les classes de hauteur. Dans la pratique, si nous appliquons directement $A\langle -1,4 \rangle$ à la note  $\writechord{A}_3$ qui correspond au La 440Hz et à la note MIDI $53$, on obtient $A\langle -1,4 \rangle(\writechord{A}_3) = A\langle -1,4 \rangle(57) = -53 = \writechord{G}_{-5}$, qui est bien trop grave pour être audible. 

Afin d'éviter de sortir de la tessiture de l'instrument, ou même du spectre auditif, nous allons restreindre l'écart entre la note initiale et son image par une transformation de gamme quelconque $T : \mathbb{Z}\rightarrow \mathbb{Z}$. On souhaite alors définir à partir de $T$ une nouvelle transformation $T\langle \beta^-, \beta^+\rangle : \mathbb{Z}\rightarrow \mathbb{Z}$ telle que $ - \beta^- \leq T\langle \beta^-, \beta^+\rangle(n) \leq \beta ^+$, où $\beta^+$ (resp. $\beta^-$) est l'interval montant (resp. descendant) maximum entre la note initiale et son image.

Posons $\beta = \beta^- + \beta^+$. Dans la plupart des cas on souhaite que $\beta = b = 12$, mais il peut être intéressant de choisir par exemple $\beta  = 2b$, pour préserver le caractère montant ou descendant d'une ligne mélodique.

Soit $r = |T(n) - n | \mod \beta$ le reste de la division euclidienne de  $|T(n) - n |$ par $\beta$. On pose alors 
$$
T\langle \beta^+, \beta^- \rangle : n \mapsto \begin{cases}
  n + r & \text{si $r \leq \beta^+$}\\
  n + r - \beta & \text{sinon}
\end{cases}
$$

Nous pouvons maintenant appliquer cette restriction de l'intervalle de hauteur à nos fonctions affines. On obtient alors un sous-ensemble contraint de transformations de gamme de la forme $A\langle \mu, \tau,\beta^-, \beta^+\rangle$. Ce sont exactement ces transformations qui sont implémentées dans Live Scaler.
  


\section{Implémentation des transformations affines : LiveScaler}
Nous allons à présent nous intéresser à l'implémentation des transformations affines présentées au début de cet article pour piloter en live l'harmonie d'un morceau de musique électronique. 

\subsection{Architecture de LiveScaler}

\begin{figure}[htbp]
  \centering
  \includegraphics[width=\textwidth]{Figures/architecture-LS.pdf}
  \caption{Architecture de Live Scaler\label{Fig:archi}}
\end{figure}

LiveScaler fonctionne à la manière d'un orchestre dont le DJ serait le chef. Chaque piste MIDI contenant un instrument virtuel (synthétiseur, sampleur, etc.) est un instrumentiste de l'orchestre. On souhaite que sur un geste du DJ, chaque instrument virtuel interprète différemment sa partition, c'est à dire le flux MIDI qu'il reçoît. Dans le cadre de LiveScaler, le DJ envoie les paramètres d'une transformation affine à tous les instruments simultanément et ceux-ci doivent appliquer cette transformation dès qu'ils la reçoivent.

L'implémentation de LiveScaler est donc séparée en deux outils interdépendants : 
\begin{enumerate}
  \item une interface (appelée \emph{Conductor}, en référence à l'analogie avec l'orchestre) qui récupère les entrées de l'utilisateur (ici le DJ) et les convertit en paramètres d'une transformation affine et envoie ces paramètres à toutes les insances de \emph{Instrument}.
  \item un plug-in MIDI appelé \emph{Instrument} qui transforme le flux MIDI entrant en appliquant à toutes les notes la transformation affine dont les paramètres ont été reçus de \emph{Conductor}.
\end{enumerate}
La Figure \ref{Fig:archi} illustre l'architecture globale de LiveScaler.

On distingue également les paramètres \emph{locaux}, qui sont propres à chaque instrument, et les paramètres \emph{globaux}, qui sont reçus du chef d'orchestre et donc communs à tous les instruments. Dans le cadre des transformations affines, les paramètres $\mu$,$\tau$, $\alpha$  et $b$ sont globaux, il correspondent dans une certaine mesure à l'harmonie actuelle du morceau. Quant à $\beta^-$ et $\beta^+$, ils sont propres à chaque instrument et s'adaptent à sa tessiture.


\subsection{Quand appliquer les transformations ? }
Lorsque \emph{Instrument} reçoit la commande d'appliquer une nouvelle transformation de gamme, il est sensé l'appliquer instantanément. Lorsque l'instrument n'est pas en train de jouer, cela ne pose aucune difficulté : il appliquera la transformation au prochain message MIDI qu'il recevra. Il se peut en revanche que l'instrument soit déjà en train de jouer une note. LiveScaler propose $3$ manières de réagir dans une telle situation. 


\begin{enumerate}
  \item \emph{Stop} : toutes les notes en train d'être jouées sont instantanément arrêtées : on envoie un message Note-off pour chaque note en cours. L'instrument reprendra son jeu, en appliquant la nouvelle transformation, au prochain message MIDI qu'il recevra. Cette option est particulièrement adaptée aux instruments dont la durée des notes est courte. 
  \item \emph{Legato} : chaque note en cours est stoppée et instantanément remplacée par son image par la nouvelle transformation. Si l'instrument virtuel est paramétré sur \emph{Legato}, alors les changements de gamme déclencheront des legatos. C'est l'option retenue par \cite{Livingstone_Muhlberger_Brown_Thompson_2010}.
  \item \emph{ReTrigger} : agit sur le même principe que \emph{Legato} à la différence  qu'un court délai est introduit entre la fin de la note en cours et la note transformée, forçant une nouvelle attaque, même si l'instrument virtuel est en mode legato.
  \item \emph{Wait} : les notes en cours continuent d'être jouées telles quelles. Si elles ne se sont pas arrêtées avant, elles seront stoppée lorsque la prochaine note sera jouée.
\end{enumerate}

Le choix entre ces trois modes se fait de manière locale, deux instances de \emph{Instrument} pourront donc réagir différemment.

\subsection{Implémentation avec Max for Live}

L'objectif principal de l'implémentation proposée était de pouvoir expérimenter le plus rapidement possible sur les transformations en tant que musicienne. Max for Live est une intégration du  langage de progammation graphique Max MSP à Ableton Live. On peut aisément communiquer avec les différences instances du logiciel, ce qui permet dans LiveScaler à \emph{Conductor} de contrôler les \emph{Instruments} avec une faible latence\footnote{En moyenne, LiveScaler introduit une latence inférieure à $1$ ms.}. La station audionumérique Ableton Live étant particulièrement populaire pour composer et produire de la musique dans le milieu de l'EDM, c'est donc naturellement que nous avons choisi Max for Live pour une première implémentation. 

La Figure \ref{fig:LiveScalerUI} montre l'interface graphique de LiveScaler. L'implémentation actuelle permet d'appliquer les transformations affines et les transformations périodiques sur un intervalle. Pour ces dernières, il suffit de renseigner manuellement l'image de chaque note de l'intervalle considéré dans un fichier pour y avoir ensuite accès via LiveScaler.












\section{Escape : une performance avec LiveScaler }
Afin de tester la pertinence des transformations proposées, j'ai utilisé LiveScaler pour créer \emph{Escape}, une performance live de trance psychédélique. J'ai notamment pu proposer cette performance en publique pour la soutenance de mon mémoire de master en Septembre 2019. Dans cette section, j'expliquerai la manière dont j'ai procédé pour cette mise en pratique et je donnerai mes impressions subjectives en tant qu'utilisatrice de LiveScaler.
\subsection{Contrôle live dans Escape}
\begin{figure}[htbp]
  \centering
  \includegraphics[width=\textwidth]{Figures/IMGP9899.jpg}
  \caption{Contrôleurs MIDI utilisés pour la performance live : à gauche le Push 2 par Ableton (contrôle de la structure du morceau) et à droite ATOM par Presonus (contrôle de l'harmonie du morceau).\label{fig:controleurs}}
\end{figure}
Pour \emph{Escape}, j'utilise deux contrôleurs MIDI distincts (voir Figure \ref{fig:controleurs}) :
\begin{itemize}
  \item le Push 2 par Ableton, qui est conçu spécialement pour contrôler Ableton Live. Il me permet de recréer en live la structure d'Escape en déclenchant ses différentes parties.
  \item l'ATOM de Presonus, composé d'une grille de$4\times 4$ touches qui me permettent de déclencher les transformations de gamme de \emph{LiveScaler}.
\end{itemize}

La figure \ref{fig:mapping-ATOM} illustre la manière (le plus souvent appelée \emph{mapping}) dont les touches de l'ATOM sont associées à des transformations de LiveScaler





\begin{figure}[htbp]
  \centering
  \includegraphics[width=\columnwidth]{Figures/Pads-config.pdf}
  \caption{Mapping de LiveScaler sur un contrôleur MIDI à $4\times 4$ touches\label{fig:mapping-ATOM}}
\end{figure}

La figure \ref{fig:mapping-ATOM} illustre la manière (le plus souvent appelée \emph{mapping}) dont les touches de l'ATOM sont associées à des transformations de LiveScaler.

Voici le détail de l'action des différentes touches : 
\begin{itemize}
  \item \LSI, \LSvi, \LSIV, \LSII, \LSV, \LSiii, \LSII, \LSvii :  les deux colones centrales déclenchent instantanément les transformations décrites précédemment. Elles sont organisées par relatives mineures/majeures.
  \item \texttt{Hist} : LiveScaler garde en mémoire un court historique des transformation précédemment appliquées. Appuyer sur \texttt{Hist} permet de déclencher une des transformations de cet historique. Des combinaisons de la touche \texttt{Hist} et des touches \texttt{Hist}, \LSMm, $2$, $3$ et $4$ permettent de naviguer dans cet historique \footnote{Pour plus de détail, se référer au manuel de LiveScaler}. 
  \item \LSpp, \LSmm : en combinant la touche $++$ (resp. $--$) avec une des transformations des colonnes centrales, on transpose cette transformation d'un demi-ton vers le haut (resp. vers le bas). 
  \item  \LSMm : permet de passer d'une transposition à une inversion et réciproquement. En pratique, combiner \LSMm avec \LSI $~$ (resp. \LSvi, \LSIV,\LSII, \LSV, \LSiii, \LSII, \LSvii) donnera la transformation \LSI $~$ (resp. \LSvi, \LSIV, \LSII, \LSV, \LSiii, \LSII, \LSvii).
  \item \LStwo, \LSthree, \LSfour : combiner une de ces touches avec une des transformations des colonnes centrales permettent de modifier le coefficient modal de cette transformation.
  \item \LSMod : en combinant \LSMod avec une des transformations, on indique à LiveScaler qu'on souhaite moduler l'harmonie de notre morceau vers cette nouvelle gamme. 
\end{itemize}

Ainsi, à partir d'une instrumentation jouant de manière répétée sur un accord de \writechord{C} majeur, on pourrait imaginer d'harmoniser en live cette instrumentation. Par exemple, si on veut reproduire la suite d'accord de la chanson Summer Nights de la comédie musicale Grease \footnote{Si, comme pour moi, cette chanson à tendance à rester dans votre tête, je suis (presque) désolée} en commençant par indiquer à LiveScaler qu'on est dans une tonalité de Ré majeur (\LSMod + \LSII). Puis, une fois l'instrumentation lancée, on apuiera tous les deux temps sur, successivement \LSI ; \LSIV ; \LSV ; \LSIV.

Puis, arrivés au moment tant attendu de la modulation d'un demi-ton vers le haut,

\begin{tabular}{ccccc}
\LSI; & \LSIV; & \LSV; & \LSpp + \LSMm +  \LSvi; & \LSMod + \LSpp + \LSI
\end{tabular}
%$$I ; IV; V; \LSpp + (M\leftrightarrow m) vi; Mod + \LSpp + I$$

pour repartir joyeusement sur \LSI ; \LSIV ; \LSV ; \LSIV, mais cette fois dans une tonalité de Mi bémol majeur. On aura ainsi obtenu la progression harmonique suivante : $$\dots - D - G - A - G - D - G - A - B\flat - E\flat -A\flat - B\flat - A\flat - \dots$$

\begin{comment}

L'objectif est de pouvoir communiquer une transformation affine via un contrôleur MIDI en un seul geste musical.


\begin{enumerate}
  \item Les distances sur le tonnetz sont similaires aux distances entre les pads
  \item Possibilité d'utilisation mélodique (on peut aisément jouer une gamme)
  \item Ancre : transposition et modulations (ajouter au cadre théorique ?)
  \item Historique : revenir aisément en arrière
\end{enumerate}
\end{comment}


\subsection{Processus de composition}
\begin{comment}
  Liste des contraintes :
\begin{enumerate}
  \item Toutes les pistes doivent être MIDI : une piste audio ne serait pas impactée par les changements de gamme
  \item Pauvreté harmonique : le morceau doit être relativement pauvre harmoniquement, idéalement rester sur le même accord tout le long.
\end{enumerate}
Préparation d'une session pour utiliser Live Scaler
\begin{enumerate}
  \item Passer toutes les pistes en MIDI (utiliser des sampleurs pour les pistes audio )
  \item S'assurer qu'on ne sort pas de la tessiture des instruments
  \item Mettre les synthétiseurs et sampleurs en mode legato au maximum.
\end{enumerate}
\end{comment}


J'ai composé \emph{Escape} dans le but de le jouer avec \emph{LiveScaler}. C'est un morceau de trance psychédélique\footnote{La trance psychédélique est souvent  appelée \emph{psytrance}, le lectorat curieux pourra écouter par exemple l'album \emph{The Gathering} (1999) du duo israëlien \emph{Infected Mushroom}.} composé  : 
\begin{itemize}
  \item d'une mélodie minimaliste durant 2 mesures jouée par un synthétiseur proche dont le son se rapproche d'un métallophone éthéré
  \item d'une ligne de basse composé de la classique \emph{rolling bass} devenue la signature de la \emph{psytrance}, et de  plusieurs autres synthétiseurs sur une unique note pédale mais aux textures sonores riches
  \item d'une rythmique séquencée à l'avance indépendante de LiveScaler (pistes audio)
  \item de quelques effets sonores (\emph{risers},\emph{downshifters}, etc.) eux aussi typiques de la \emph{psytrance}, que je déclenche ponctuellement à l'aide du Push.
\end{itemize}

L'objectif était d'une part de partir d'un morceau simple, basé sur un unique accord (ici La mineur) et d'enrichir son harmonie et sa ligne mélodique avec \emph{LiveScaler}; et d'autre part de proposer un morceau typique d'un genre de musique électronique populaire extrêmement codifié (ici la \emph{psytrance}). Le choix du genre n'est pas anodin, il correspond à un goût personnel mais je souhaitais aussi utiliser un outil expérimental pour un morceu de musique qui, lui, n'a rien d'expérimentale. Pour moi il s'agit plus avec LiveScaler d'explorer de nouvelles modalités live que de nouveaux horizons musicaux.

\subsection{Retour d'expérience}
\begin{comment}
  \begin{enumerate}
  \item Anticiper ses gestes : la transformation de gamme doit être prise en compte avant le moment précis où l'on souhaite qu'elle arrive. Peut devenir très technique assez rapidement
  \item Il est possible de faire concorder le geste avec le changement de gamme (malgré la latence et la faible avance nécessaire) : avantage par rapport au public
  \item Live Scaler est particulièrement adapté à la psytrance et à l'EDM, qui sont initialement harmoniquement pauvre.
\end{enumerate}
\end{comment}



\section{Travaux connexes}

\subsection{Théorie transformationnelle}
\begin{comment}
\begin{enumerate}
  \item Les transformations affines sont une généralisation 
  \begin{enumerate}
    \item des Tonnetz
    \item des k-Nets
  \end{enumerate}
  \item ces transformations sont étudiées et utilisées surtout dans un contexte d'analyse et d'aide à la composition (OpenMusic)
\end{enumerate}
\end{comment}

La théorie transformationnelle (pour une introduction généraliste du point de vue mathématique, voir \cite{andreatta2008calcul}, et du point de vue musicologique, lire \cite{andreatta2014introduction}) prend ses racines dans la \emph{Set-Theory}, qui se concentre sur la notion de classe de hauteur, c'est à dire un ensemble de notes identiques à l'octave près (\cite{forte1973structure}). Elle propose une approche plus algébrique que la \emph{Set-Theory}, en se concentrant, entre autre sur la notion de transformation entre ensembles de classes de hauteurs (\cite{lewin1987generalized}).

Les transformations affines sont directement inspirées de la théorie transformationnelle, et plus particulièrement des automorphismes du groupe T/I présentés par \textcite{lewin1990klumpenhouwer}. L'unique différence avec ceux-ci est que les transformations affines ne sont pas nécessairement bijectives et autorisent donc un coefficient modal qui ne soit pas nécessairement premier avec $12$. Les transformations affines sont donc une vision plus terre à terre des automorphismes proposés par Lewin et Klumpenhouwer, tout en proposant quelques transformations supplémentaires sortant du système tonal.

Plusieurs outils s'appuient plus ou moins explicitement sur les représentations de la théorie transformationnelle, en particulier OpenMusic, qui propose une aide à la composition directement inspirée de cell-ci (\cite{andreatta2003implementing}, \cite{andreatta2003formalisation}). Depuis une dizaine d'année, OpenMusic essaie de concilier l'approche hors du temps (approche guidée par les demandes) de l'aide à la composition avec l'approche temps-réel propre à la performance (\cite{bresson2014reactive}, \cite{bresson2017next}). Plus récemment, Bach propose lui aussi cette approche hybride mais en partant de Max MSP, un language de programmation fondamentalement temps-réel et guidé par les données (\cite{agostini2021programming}).

Les outils évoqués ci-dessus offrent une grande flexibilité pour appliquer des transformations potentiellement bien plus sophistiquées que les transformations affines, et sont a priori tous capables de le faire en live. Pour autant, cela demanderait un grand travail préparatoire de programmation et de mapping avant d'arriver à un résultat fluide. C'est exactement ce travail qui est fait par LiveScaler, mais sur un nombre restreint de transformations, ici jugées pertinentes. LiveScaler sacrifie donc la flexibilité dans le choix des transformations au profit d'une utilisation immédiate, et sans connaissances de progammation requises, pour faire de la musique live.

\subsection{Live Coding}

Une approche intermédiaire offrant plus de flexibilité, mais moins d'immédiateté est celle du \emph{live coding}. Pendant une performance de \emph{live coding}, le musicien utilise un langage de programmation dédié pour coder en live un morceau de musique (\cite{blackwell2022live}). Tidal, proposé par \textcite{mclean2010tidal} permet de manipuler et transformer des motifs en live. Il reprend notamment les transformations de \textcite{spiegel1981manipulations} et plus particulièrement les transpositions et inversions (qui correspondent aux coefficients modaux $1$ et $-1$ dans le paradigme des transformations affines proposé ici).

Si une des revendications initialement associées à cette pratique était de s'affranchir des contraintes et rigidités des stations audionumériques telle qu'Ableton Live (\cite{collins2003live}), l'ajout de langages de scripting  ainsi que la posibilité de contrôler les stations numériques \footnote{On peut par exemple piloter  Ableton Live et FL Studio avec Python, Logic et Bitwig avec JavaScript.} avec des langages de programmation graphique haut niveau \footnote{On trouve dans les stations audionumériques commerciales de plus en plus de langages de "\emph{patching}" permettant de contrôler le logiciel ou de créer des plugins audio de manière modulaire : voir par exemple Max for Live pour Ableton Live, FL FlowStone pour FL Studio, The Grid pour Bitwig.} semblent avoir développé leur usage dans les pratiques de musique algorithmique live (\cite{collins2014algorave}). Ces pratiques, dans lesquelles s'incrit le présent article, permettent de combiner d'une part la flexibilité et la liberté qu'offrent un langage de programmation et d'autre part l'accès aux outils de production commerciaux utilisés par l'industrie de la musique. C'est particulièrement important dans le cadre de l'EDM\footnote{\emph{Electronic Dance Music}, un terme parapluie regroupant de nombreux genres de musique électronique tels que la house, la techno, la trance, la drum n bass, le dubstep, etc. }, qui utilise intensivement ces outils de production sophistiqués (\cite{fraser2012spaces}).


\subsection{Macro contrôle d'un morceau de musique électronique}
\begin{enumerate}
  \item "Changing musical emotion"
\end{enumerate}






\section{Conclusion}
\section{Conclusion}

Tous les outils cités plus haut offrent une grande flexibilité pour appliquer des transformations potentiellement bien plus sophistiquées que les transformatiions affines, et sont a priori tous capables de le faire en live. Une spécificité de LiveScaler est l'idée de d'utiliser ces transformations comme un instrument de musique, et de les rendre directement accessibles aux musiciens même s'ils n'ont aucune affinité avec la programmation.

Utiliser la flexibilité et la rapidité de bell (\cite{agostini2020programmer}). Utiliser un meilleur protocole que MIDI (OSC,MP \cite{goudard2017mapping})
\subsection{Perpectives}
\begin{enumerate}
  \item Associer les changements de gammes à un contrôle d'image (VJing)
  \item Proposer un macro contrôle pour le rythme
  \item Obtenir le retour utilisateurs de musiciens et performeurs
\end{enumerate}

\subsection{Remerciements}
Je tenais à remercier ici David Janin et Martin Laliberté, mes encadrants de thèse, pour leur relecture et leurs nombreux conseils. Merci également à Chloé Lavrat, pour avoir pris le temps de me lire et de nos nombreuses discussions sur le sujet, toujours très inspirantes.

\newpage
%\printbibliography

\bibliography{refs} 
\bibliographystyle{plain}


\end{document}
