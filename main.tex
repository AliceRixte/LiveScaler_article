\documentclass[french,11pt]{article}
\usepackage[utf8]{inputenc}
\usepackage[T1]{fontenc}
\usepackage{babel}
\usepackage{eurosym}
\usepackage[affil-it]{authblk}

\usepackage[table]{xcolor}

\usepackage{tikz}
\usetikzlibrary{arrows,decorations.markings}
\usetikzlibrary{cd}
\usetikzlibrary{shapes.geometric,fit}
\usetikzlibrary{positioning}



\usepackage[top=4cm, bottom=4cm, left=3cm, right=3cm]{geometry}
\usepackage[fencedCode,inlineFootnotes,citations,definitionLists,hashEnumerators,smartEllipses,hybrid,pipeTables, tableCaptions]{markdown}
\usepackage{keyval}

\usepackage[backend=biber,style=alphabetic,citestyle=authoryear-comp]{biblatex}
\addbibresource{refs.bib}

\usepackage{minted}
\setkeys{Gin}{width=\linewidth,totalheight=\textheight,keepaspectratio}

\usepackage{mathtools}
\usepackage{amsmath}
\usepackage{amsfonts}
\usepackage{amsthm}
\usepackage{amssymb}



\usepackage{graphicx}
\graphicspath{{img/}}
\usepackage{wrapfig}
\usepackage{float}

\usepackage{hyperref}
\interfootnotelinepenalty=10000

\usepackage{setspace}
\onehalfspacing

\usepackage{caption}
\usepackage{subcaption}
\tikzcdset{every label/.append style = {font = \tiny}}

\usepackage{csquotes}
\usepackage{comment}


\usepackage{tabularx}
\definecolor{tableShade}{gray}{0.9}

\usepackage[shortlabels]{enumitem}





\title{LiveScaler}
\author{Alice Rixte}
\begin{document}
\maketitle

\section{Introduction}

\section{Transformations de gammes}

\begin{comment}
  \subsection{Quelles transformations de gamme autoriser ?}
Les transformation affines de la forme $an + b$ où $n$ est la note de départ.
\subsection{Pourquoi ces transformations ?}
\begin{enumerate}
  \item Ces transformations sont adaptées à la musique tonale occidentale : passage du majeur au mineur.
  \item Elles sont facilement implémentables
  \item Elles peuvent être exprimées par une paire d'entiers, ce qui permet de les communiquer directement via MIDI.
\end{enumerate}


\subsection{Peut-on quand même sortir de la tonalité ?}
Oui : \begin{enumerate}
  \item on peut passer d'une gamme quelconque à une gamme par tons
  \item elles s'appliquent dans un contexte microtonal
  \item Possibilité d'ajouter une permutation quelconque personnalisée
\end{enumerate} 
\end{comment}


Nous définissons ici une transformation de gamme comme une fonction qui à toute hauteur de note associe une nouvelle hauteur de note a priori quelconque, autrement dit une fonction $\mathbb{Z}$ dans $\mathbb{Z}$. Un bon exemple de transformation de gamme est la transposition : à chaque note $n$ on associe la note $n+k$ (resp. $n-k$) décalée de $k$ demi-tons vers le haut (resp. vers le bas). Une transposition de $t$ demi-tons est représentée par la transformation de gamme $T_t : n \mapsto n+t$.

Les transformations de gamme qui vont nous intéresser ici sont les transformation affines, c'est-à-dire les fonctions de la forme $A_{m,t} : n \mapsto mn + t$ avec $m$ le \emph{coefficient multiplicateur} de la transformation affine et $t$ le \emph{coefficient de transposition}. Par exemple, la transposition de $3$ demi-tons vers le haut correspond par exemple à la transformation affine $A_{1,3}$. De manière plus générale, les transpositions sont de la forme $A_{1,t}$ avec $t$ l'intervalle de transposition (positif lorsque l'on transpose vers le haut et négatif lorsque l'on transpose vers le bas).

Notre choix de nous restreindre aux transformations affines repose sur le fait d'une part qu'on peut très facilement les implémenter en les appliquant à un flux MIDI et d'autre part sur leur expressivité : elles permettent par exemple d'exprimer les transpositions, de passer d'une gamme mineure à une gamme majeure (et réciproquement), ou encore de passer de n'importe quelle gamme diatonique à une gamme par ton. 

\subsection{Inversion}


% https://q.uiver.app/?q=WzAsNTQsWzUsMSwiQ197MH0iXSxbNiwxLCJDXFwjX3swfSJdLFs3LDEsIkRfMCJdLFs4LDEsIkRcXCNfMCJdLFs5LDEsIkVfMCJdLFsxMCwxLCJGXzAiXSxbMTEsMSwiRlxcI18wIl0sWzEyLDEsIkdfMCJdLFsxMywxLCIuLi4iXSxbOCwwXSxbNSwyLCIwIl0sWzYsMiwiMSJdLFs3LDIsIjIiXSxbOCwyLCIzIl0sWzksMiwiNCJdLFsxMCwyLCI1Il0sWzExLDIsIjYiXSxbMTIsMiwiNyJdLFs3LDNdLFs0LDEsIkJfey0xfSJdLFszLDEsIkFcXCNfey0xfSJdLFsyLDEsIkFfey0xfSJdLFsyLDIsIi0zIl0sWzMsMiwiLTIiXSxbNCwyLCItMSJdLFsxLDIsIi4uLiJdLFs5LDQsIjQiXSxbMiw0LCItMyJdLFs3LDQsIjIiXSxbNSw0LCIwIl0sWzEwLDQsIjUiXSxbMTIsNCwiNyJdLFs1LDUsIkNfMCJdLFs0LDQsIi0xIl0sWzMsNCwiLTIiXSxbNiw0LCIxIl0sWzgsNCwiMyJdLFsxMSw0LCI2Il0sWzIsNSwiQV97LTF9Il0sWzMsNSwiQVxcI197LTF9Il0sWzQsNSwiQl97LTF9Il0sWzEsNSwiLi4uIl0sWzEsMSwiLi4uIl0sWzEzLDIsIi4uLiJdLFsxMyw0LCIuLi4iXSxbMTMsNSwiLi4uIl0sWzYsNSwiQ1xcI18wIl0sWzcsNSwiRF8wIl0sWzgsNSwiRFxcI18wIl0sWzksNSwiRV8wIl0sWzEwLDUsIkZfMCJdLFsxMSw1LCJGXFwjXzAiXSxbMTIsNSwiR18wIl0sWzAsM10sWzEwLDI2XSxbMTIsMjgsIiIsMCx7ImNvbG91ciI6WzAsMCw0Nl0sInN0eWxlIjp7ImJvZHkiOnsibmFtZSI6ImRhc2hlZCJ9fX1dLFsxNCwyOV0sWzI0LDMwLCIiLDIseyJjb2xvdXIiOlswLDAsNDZdLCJzdHlsZSI6eyJib2R5Ijp7Im5hbWUiOiJkYXNoZWQifX19XSxbMjIsMzFdLFsxNSwzMywiIiwyLHsiY29sb3VyIjpbMCwwLDQ2XSwic3R5bGUiOnsiYm9keSI6eyJuYW1lIjoiZGFzaGVkIn19fV0sWzE3LDI3XV0=






\begin{figure}[ht]
  \centering
    \caption{La transformation $A_{-1,4}$  : passage à la relative mineure de $C$ à $Am$}\label{fig:inner-proof}
        \begin{tikzpicture}[baseline= (a).base]    

          \node[scale=1.3] (a) at (0,0){
        \begin{tikzcd}[column sep = tiny]
          % https://q.uiver.app/?q=WzAsNTQsWzUsMSwiQ197MH0iXSxbNiwxLCJDXFwjX3swfSJdLFs3LDEsIkRfMCJdLFs4LDEsIkRcXCNfMCJdLFs5LDEsIkVfMCJdLFsxMCwxLCJGXzAiXSxbMTEsMSwiRlxcI18wIl0sWzEyLDEsIkdfMCJdLFsxMywxLCIuLi4iXSxbOCwwXSxbNSwyLCIwIl0sWzYsMiwiMSJdLFs3LDIsIjIiXSxbOCwyLCIzIl0sWzksMiwiNCJdLFsxMCwyLCI1Il0sWzExLDIsIjYiXSxbMTIsMiwiNyJdLFs3LDNdLFs0LDEsIkJfey0xfSJdLFszLDEsIkFcXCNfey0xfSJdLFsyLDEsIkFfey0xfSJdLFsyLDIsIi0zIl0sWzMsMiwiLTIiXSxbNCwyLCItMSJdLFsxLDIsIi4uLiJdLFs5LDQsIjQiXSxbMiw0LCItMyJdLFs3LDQsIjIiXSxbNSw0LCIwIl0sWzEwLDQsIjUiXSxbMTIsNCwiNyJdLFs1LDUsIkNfMCJdLFs0LDQsIi0xIl0sWzMsNCwiLTIiXSxbNiw0LCIxIl0sWzgsNCwiMyJdLFsxMSw0LCI2Il0sWzIsNSwiQV97LTF9Il0sWzMsNSwiQVxcI197LTF9Il0sWzQsNSwiQl97LTF9Il0sWzEsNSwiLi4uIl0sWzEsMSwiLi4uIl0sWzEzLDIsIi4uLiJdLFsxMyw0LCIuLi4iXSxbMTMsNSwiLi4uIl0sWzYsNSwiQ1xcI18wIl0sWzcsNSwiRF8wIl0sWzgsNSwiRFxcI18wIl0sWzksNSwiRV8wIl0sWzEwLDUsIkZfMCJdLFsxMSw1LCJGXFwjXzAiXSxbMTIsNSwiR18wIl0sWzAsM10sWzEwLDI2XSxbMTIsMjgsIiIsMCx7ImNvbG91ciI6WzAsMCw0Nl0sInN0eWxlIjp7ImJvZHkiOnsibmFtZSI6ImRhc2hlZCJ9fX1dLFsxNCwyOV0sWzI0LDMwLCIiLDIseyJjb2xvdXIiOlswLDAsNDZdLCJzdHlsZSI6eyJib2R5Ijp7Im5hbWUiOiJkYXNoZWQifX19XSxbMjIsMzFdLFsxNSwzMywiIiwyLHsiY29sb3VyIjpbMCwwLDQ2XSwic3R5bGUiOnsiYm9keSI6eyJuYW1lIjoiZGFzaGVkIn19fV0sWzE3LDI3XV0=
	& {...} & {A_{-1}} & {A\#_{-1}} & {B_{-1}} & {C_{0}} & {C\#_{0}} & {D_0} & {D\#_0} & {E_0} & {F_0} & {F\#_0} & {G_0} & {...} \\
	& {...} & {-3} & {-2} & {-1} & 0 & 1 & 2 & 3 & 4 & 5 & 6 & 7 & {...} \\
	{} &&&&&&& {} \\
	&& {-3} & {-2} & {-1} & 0 & 1 & 2 & 3 & 4 & 5 & 6 & 7 & {...} \\
	& {...} & {A_{-1}} & {A\#_{-1}} & {B_{-1}} & {C_0} & {C\#_0} & {D_0} & {D\#_0} & {E_0} & {F_0} & {F\#_0} & {G_0} & {...}
	\arrow[from=2-6, to=4-10]
	\arrow[color={rgb,255:red,117;green,117;blue,117}, dashed, from=2-8, to=4-8]
	\arrow[from=2-10, to=4-6]
	\arrow[color={rgb,255:red,117;green,117;blue,117}, dashed, from=2-5, to=4-11]
	\arrow[from=2-3, to=4-13]
	\arrow[color={rgb,255:red,117;green,117;blue,117}, dashed, from=2-11, to=4-5]
	\arrow[from=2-13, to=4-3]
\end{tikzcd}
    };
      \end{tikzpicture}   
\end{figure}







\section{Implémentation des transformations de gamme : l'outil LiveScaler}
\subsection{Justification du choix M4L /Ableton Live}
\begin{enumerate}
  \item Outil populaire dans le domaine de l'EDM
  \item Facilité d'utilisation et intégration dans un environnement 
  \item Prototypage aisé dans M4L
  \item Communication entre les entités d'Ableton Live via messages M4L
\end{enumerate}
\subsection{Comment appliquer la transformation ?}
Architecture chef d'orchestre/ instrument : 
\begin{enumerate}
  \item Un seul changement de gamme est envoyé à tous les instruments simultanément
  \item Chaque instrument applique la transformation selon ses propres paramètres (e.g. intervalle de repliement)
  \item Deux types de paramètres : \begin{itemize}
    \item Paramètres globaux envoyés sous la forme de messages à tous les instruments sous forme de message.
    \item Paramètre propre à chaque instrument d'interprétation des messages reçus.
    \end{itemize}
\end{enumerate}

\subsection{Quand appliquer la transformation ? }
Instantanément, car le flux MIDI en entrée fait office de "quantize".  ===> Quid des notes en cours ?
\begin{enumerate}
  \item On les arrête
  \item Legato
  \item Retrigger
\end{enumerate}
===> Le choix se fait séparément pour chaque instrument

\subsection{Comment éviter que les notes soient trop aiguës ou trop graves ?}
Repliement des transformation de gamme sur un intervalle
\begin{enumerate}
  \item Limiter l'écart de hauteur entre la note initiale et la note après la transformation.
  \item Possibilité d'adapter le repliement au morceau
\end{enumerate}
===> le choix se fait séparément pour chaque instrument
\section{Performance live avec LiveScaler}

\section{Travaux connexes}
\subsection{Transformations de gamme en live coding}
\begin{enumerate}
  \item propositions de Tidal
  \item algorave
\end{enumerate}
\subsection{Théorie transformationnelle}
\begin{enumerate}
  \item Les transformations affines sont une généralisation 
  \begin{enumerate}
    \item des Tonnetz
    \item des k-Nets
  \end{enumerate}
  \item ces transformations sont étudiées et utilisées surtout dans un contexte d'analyse et d'aide à la composition (OpenMusic)
\end{enumerate}

\subsection{Macro contrôle d'un morceau de musique électronique}
\begin{enumerate}
  \item "Changing musical emotion"
\end{enumerate}

\section{Conclusion}

\section{Annexe : mapping de LiveScaler}

\newpage
\printbibliography

\end{document}
