\documentclass[french,11pt]{article}
\usepackage[utf8]{inputenc}
\usepackage[T1]{fontenc}
\usepackage{babel}
\usepackage{eurosym}
\usepackage[affil-it]{authblk}

\usepackage[table]{xcolor}

\usepackage{tikz}
\usetikzlibrary{arrows,decorations.markings}
\usetikzlibrary{cd}
\usetikzlibrary{shapes.geometric,fit}
\usetikzlibrary{positioning}

\usepackage{leadsheets}

\usepackage[top=4cm, bottom=4cm, left=3cm, right=3cm]{geometry}
\usepackage[fencedCode,inlineFootnotes,citations,definitionLists,hashEnumerators,smartEllipses,hybrid,pipeTables, tableCaptions]{markdown}
\usepackage{keyval}

\usepackage[backend=biber,style=alphabetic,citestyle=authoryear-comp]{biblatex}
\addbibresource{refs.bib}

\usepackage{minted}
\setkeys{Gin}{width=\linewidth,totalheight=\textheight,keepaspectratio}

\usepackage{mathtools}
\usepackage{amsmath}
\usepackage{amsfonts}
\usepackage{amsthm}
\usepackage{amssymb}



\usepackage{graphicx}
\graphicspath{{img/}}
\usepackage{wrapfig}
\usepackage{float}

\usepackage{hyperref}
\interfootnotelinepenalty=10000

\usepackage{setspace}
\onehalfspacing

\usepackage{caption}
\usepackage{subcaption}
\tikzcdset{every label/.append style = {font = \tiny}}

\usepackage{csquotes}
\usepackage{comment}


\usepackage{tabularx}
\definecolor{tableShade}{gray}{0.9}

\usepackage[shortlabels]{enumitem}





\title{LiveScaler}
\author{Alice Rixte}
\begin{document}
\maketitle

\tableofcontents

\section{Introduction}

\section{Transformations de gammes}
\begin{comment}
  \subsection{Quelles transformations de gamme autoriser ?}
Les transformation affines de la forme $an + b$ où $n$ est la note de départ.
\subsection{Pourquoi ces transformations ?}
\begin{enumerate}
  \item Ces transformations sont adaptées à la musique tonale occidentale : passage du majeur au mineur.
  \item Elles sont facilement implémentables
  \item Elles peuvent être exprimées par une paire d'entiers, ce qui permet de les communiquer directement via MIDI.
\end{enumerate}


\subsection{Peut-on quand même sortir de la tonalité ?}
Oui : \begin{enumerate}
  \item on peut passer d'une gamme quelconque à une gamme par tons
  \item elles s'appliquent dans un contexte microtonal
  \item Possibilité d'ajouter une permutation quelconque personnalisée
\end{enumerate} 
\end{comment}

Nous définissons ici une \emph{transformation de gamme} comme une fonction qui à toute hauteur de note associe une nouvelle hauteur de note a priori quelconque, autrement dit une fonction $\mathbb{Z}$ dans $\mathbb{Z}$. Un bon exemple de transformation de gamme est la transposition : à chaque note $n$ on associe la note $n+\tau$  décalée de $\tau$ demi-tons vers l'aigu lorsque $\tau$ est positif et vers le grave lorsque $\tau$ est négatif. Ainsi, une transposition de $\tau$ demi-tons est représentée par la transformation de gamme $ n \mapsto n+\tau$ (voir Figure \ref{fig:transp}).


\subsection{Transformations affines}

Les transformations de gamme qui vont nous intéresser ici sont les \emph{transformation affines}\footnote{Cette définition est inspirée par les automorphismes du groupe $T/I$ présentés par \textcite{lewin1990klumpenhouwer}.}, c'est-à-dire les fonctions de la forme $A\langle\mu,\tau\rangle : n \mapsto \mu n + \tau$ avec $\mu$ le \emph{coefficient modal} de la transformation affine et $\tau$ le \emph{coefficient de transposition}. 

Les transformations affines ont la propriété importante de préserver les classes de hauteur \footnote{En effet, pour toute base $b\in \mathbb{N}^*$, $\forall n_1,n_2 \in \mathbb{Z}, n_1 \equiv n_2 \mod b \implies \mu n_1 + \tau \equiv \mu n_2 + \tau \mod b$. Avec $b=12$, on obtient le résultat pour les classes de hauteurs dodécaphoniques. } c'est-à-dire que si deux notes sont identiques à l'octave près, alors elles le seront toujours une fois la transformation affine appliquée. 


\subsection{Exemples de transformations}
Nous allons à présent étudier plusieurs exemples afin de donner au lecteur un aperçu de leur expressivité.

\subsubsection{Transpositions}
Lorsque $\mu = 1$, les transformations affines $A\langle 1,\tau \rangle : n \mapsto n + \tau$ permettent de représenter toutes les transpositions possibles (voir Figure \ref{fig:transp}).

\begin{figure}[ht]
  \centering
  \begin{tikzpicture}[baseline= (a).base]    

    \node[scale=1] (a) at (0,0){
    \begin{tikzcd}[column sep=0pt, minimum width=11.5mm, row sep=0.1cm]
    {...} & {A_{-1}} & {A\sharp_{-1}} & {B_{-1}} & {C_{0}} & {C\sharp_{0}} & {D_0} & {D\sharp_0} & {E_0} & {F_0} & {F\sharp_0} & {G_0} & {...} \\
    {...} & {-3} & {-2} & {-1} & 0 & 1 & 2 & 3 & 4 & 5 & 6 & 7 & {...} \\
    {} &&&&&&&&&&& {} \\
    {} &&&&&&&&&&& {} \\
    {} &&&&&&&&&&& {} \\
    {} &&&&&&&&&&& {} \\
    {} &&&&&&&&&&& {} \\
    {} &&&&&&&&&&& {} \\
    {...} & {-3} & {-2} & {-1} & 0 & 1 & 2 & 3 & 4 & 5 & 6 & 7 & {...} \\
    {...} & {A_{-1}} & {A\sharp_{-1}} & {B_{-1}} & {C_0} & {C\sharp_0} & {D_0} & {D\sharp_0} & {E_0} & {F_0} & {F\sharp_0} & {G_0} & {...}
    \arrow[color={rgb,255:red,117;green,117;blue,117}, dotted, from=2-1, to=9-3]
    \arrow[from=2-2, to=9-4]
    \arrow[from=2-3, to=9-5]
    \arrow[from=2-4, to=9-6]
    \arrow[from=2-5, to=9-7]
    \arrow[from=2-6, to=9-8]
    \arrow[from=2-7, to=9-9]
    \arrow[from=2-8, to=9-10]
    \arrow[from=2-9, to=9-11]
    \arrow[from=2-10, to=9-12]
    \arrow[color={rgb,255:red,117;green,117;blue,117}, dotted, from=2-11, to=9-13]
    \end{tikzcd}
  };
  \end{tikzpicture}
  \caption{La transformation $A \langle 1,2 \rangle : n \mapsto n + 2$ correspond transposition d'un ton vers l'aigu}\label{fig:transp}
\end{figure}

\subsubsection{Inversions}
\begin{figure}[ht]
  \centering
  \begin{tikzpicture}[baseline= (a).base]    

    \node[scale=1] (a) at (0,0){
    \begin{tikzcd}[column sep=0pt, minimum width=11.5mm, row sep=0.1cm]
    % https://q.uiver.app/?q=WzAsNTQsWzUsMSwiQ197MH0iXSxbNiwxLCJDXFwjX3swfSJdLFs3LDEsIkRfMCJdLFs4LDEsIkRcXCNfMCJdLFs5LDEsIkVfMCJdLFsxMCwxLCJGXzAiXSxbMTEsMSwiRlxcI18wIl0sWzEyLDEsIkdfMCJdLFsxMywxLCIuLi4iXSxbOCwwXSxbNSwyLCIwIl0sWzYsMiwiMSJdLFs3LDIsIjIiXSxbOCwyLCIzIl0sWzksMiwiNCJdLFsxMCwyLCI1Il0sWzExLDIsIjYiXSxbMTIsMiwiNyJdLFs3LDNdLFs0LDEsIkJfey0xfSJdLFszLDEsIkFcXCNfey0xfSJdLFsyLDEsIkFfey0xfSJdLFsyLDIsIi0zIl0sWzMsMiwiLTIiXSxbNCwyLCItMSJdLFsxLDIsIi4uLiJdLFs5LDQsIjQiXSxbMiw0LCItMyJdLFs3LDQsIjIiXSxbNSw0LCIwIl0sWzEwLDQsIjUiXSxbMTIsNCwiNyJdLFs1LDUsIkNfMCJdLFs0LDQsIi0xIl0sWzMsNCwiLTIiXSxbNiw0LCIxIl0sWzgsNCwiMyJdLFsxMSw0LCI2Il0sWzIsNSwiQV97LTF9Il0sWzMsNSwiQVxcI197LTF9Il0sWzQsNSwiQl97LTF9Il0sWzEsNSwiLi4uIl0sWzEsMSwiLi4uIl0sWzEzLDIsIi4uLiJdLFsxMyw0LCIuLi4iXSxbMTMsNSwiLi4uIl0sWzYsNSwiQ1xcI18wIl0sWzcsNSwiRF8wIl0sWzgsNSwiRFxcI18wIl0sWzksNSwiRV8wIl0sWzEwLDUsIkZfMCJdLFsxMSw1LCJGXFwjXzAiXSxbMTIsNSwiR18wIl0sWzAsM10sWzEwLDI2XSxbMTIsMjgsIiIsMCx7ImNvbG91ciI6WzAsMCw0Nl0sInN0eWxlIjp7ImJvZHkiOnsibmFtZSI6ImRhc2hlZCJ9fX1dLFsxNCwyOV0sWzI0LDMwLCIiLDIseyJjb2xvdXIiOlswLDAsNDZdLCJzdHlsZSI6eyJib2R5Ijp7Im5hbWUiOiJkYXNoZWQifX19XSxbMjIsMzFdLFsxNSwzMywiIiwyLHsiY29sb3VyIjpbMCwwLDQ2XSwic3R5bGUiOnsiYm9keSI6eyJuYW1lIjoiZGFzaGVkIn19fV0sWzE3LDI3XV0=
      {...} & {A_{-1}} & {A\sharp_{-1}} & {B_{-1}} & {C_{0}} & {C\sharp_{0}} & {D_0} & {D\sharp_0} & {E_0} & {F_0} & {F\sharp_0} & {G_0} & {...} \\
      {...} & {-3} & {-2} & {-1} & 0 & 1 & 2 & 3 & 4 & 5 & 6 & 7 & {...} \\
      {} &&&&&&&&&&& {} \\
      {} &&&&&&&&&&& {} \\
      {} &&&&&&&&&&& {} \\
      {} &&&&&&&&&&& {} \\
      {} &&&&&&&&&&& {} \\
      {} &&&&&&&&&&& {} \\
      {...} & {-3} & {-2} & {-1} & 0 & 1 & 2 & 3 & 4 & 5 & 6 & 7 & {...} \\
      {...} & {A_{-1}} & {A\sharp_{-1}} & {B_{-1}} & {C_0} & {C\sharp_0} & {D_0} & {D\sharp_0} & {E_0} & {F_0} & {F\sharp_0} & {G_0} & {...}
      \arrow[from=2-5, to=9-9]
      \arrow[color={rgb,255:red,117;green,117;blue,117}, dashed, from=2-7, to=9-7]
      \arrow[from=2-9, to=9-5]
      \arrow[color={rgb,255:red,117;green,117;blue,117}, dashed, from=2-4, to=9-10]
      \arrow[from=2-2, to=9-12]
      \arrow[color={rgb,255:red,117;green,117;blue,117}, dashed, from=2-10, to=9-4]
      \arrow[from=2-12, to=9-2]
    \end{tikzcd}
    };
  \end{tikzpicture}   
  \caption{La transformation $A_{-1,4} :n\mapsto -n + 4$ correspond au passage à la relative mineure de $C$ à $Am$\label{fig:inversion}}
  \medskip
  \small
  Pour des raisons de lisibilité seules les flèches de la gamme de do majeur ont été tracées. On notera la passage de l'accord $C$ à $Am$ et de $Am$ à $C$.
\end{figure}

Lorsque $\mu = -1$, les transformations $A \langle -1,\tau\rangle : n\mapsto -n + \tau$ permettent de passer d'une gamme majeure à une gamme mineure et réciproquement. La Figure \ref{fig:inversion} illustre la manière dont le transformation $A\langle -1,4 \rangle$ envoie l'accord de Do majeur $C_0,E_0,G_0$ vers un accord de La mineur $A_{-1},C_0,E_0$ mais aussi envoie l'accord de la mineur $A_{-1},C_0,E_0$ sur l'accord de do majeur  $C_0,E_0,G_0$. Comme les transformations affines préservent les classes de hauteur, on peut considérer que $A \langle -1,4\rangle$ envoie Do majeur sur La mineur et La mineur sur Do majeur. Le tableau \ref{tab:triadesA-14} explicite l'image des accords de la gamme de Do majeur par $A\langle -1,4 \rangle$.



\begin{table}[ht]
  
  \centering % instead of \begin{center}
  \begin{tabular}{ccc}
      \writechord{Cma} & $\mapsto$ & \writechord{Ami}\\
      \writechord{Dmi} & $\mapsto$ & \writechord{Gma}\\
      \writechord{Emi} & $\mapsto$ & \writechord{Fma}\\
      \writechord{Fma} & $\mapsto$ & \writechord{Emi}\\
      \writechord{Gma} & $\mapsto$ & \writechord{Dmi}\\
      \writechord{Ami} & $\mapsto$ & \writechord{Cma}\\
      \writechord{Bo} & $\mapsto$ & \writechord{Bo}
  \end{tabular}
  \caption{Image des accords de la gamme de do majeur par $A\langle -1, 4 \rangle$\label{tab:triadesA-14}}
\end{table}

\subsection{Transformation vers une gamme par ton}
Lorsque $\mu = 2$ ou $\mu = -2$, les transformations affines envoient n'importe quelle gamme vers une gamme apparentée à une gamme par tons. Les 
\subsection{Restriction de l'écart de hauteur}

\section{Implémentation des transformations de gamme : LiveScaler}
\subsection{Justification du choix M4L /Ableton Live}
\begin{enumerate}
  \item Outil populaire dans le domaine de l'EDM
  \item Facilité d'utilisation et intégration dans un environnement 
  \item Prototypage aisé dans M4L
  \item Communication entre les entités d'Ableton Live via messages M4L
\end{enumerate}
\subsection{Comment appliquer la transformation ?}
Architecture chef d'orchestre/ instrument : 
\begin{enumerate}
  \item Un seul changement de gamme est envoyé à tous les instruments simultanément
  \item Chaque instrument applique la transformation selon ses propres paramètres (e.g. intervalle de repliement)
  \item Deux types de paramètres : \begin{itemize}
    \item Paramètres globaux envoyés sous la forme de messages à tous les instruments sous forme de message.
    \item Paramètre propre à chaque instrument d'interprétation des messages reçus.
    \end{itemize}
\end{enumerate}

\subsection{Quand appliquer la transformation ? }
Instantanément, car le flux MIDI en entrée fait office de "quantize".  ===> Quid des notes en cours ?
\begin{enumerate}
  \item On les arrête
  \item Legato
  \item Retrigger
\end{enumerate}
===> Le choix se fait séparément pour chaque instrument

\subsection{Comment éviter que les notes soient trop aiguës ou trop graves ?}
Repliement des transformation de gamme sur un intervalle
\begin{enumerate}
  \item Limiter l'écart de hauteur entre la note initiale et la note après la transformation.
  \item Possibilité d'adapter le repliement au morceau
\end{enumerate}
===> le choix se fait séparément pour chaque instrument
\section{Performance live avec LiveScaler}

\section{Travaux connexes}
\subsection{Transformations de gamme en live coding}
\begin{enumerate}
  \item propositions de Tidal
  \item algorave
\end{enumerate}
\subsection{Théorie transformationnelle}
\begin{enumerate}
  \item Les transformations affines sont une généralisation 
  \begin{enumerate}
    \item des Tonnetz
    \item des k-Nets
  \end{enumerate}
  \item ces transformations sont étudiées et utilisées surtout dans un contexte d'analyse et d'aide à la composition (OpenMusic)
\end{enumerate}

\subsection{Macro contrôle d'un morceau de musique électronique}
\begin{enumerate}
  \item "Changing musical emotion"
\end{enumerate}

\section{Conclusion}

\section{Annexe : mapping de LiveScaler}

\newpage
\printbibliography

\end{document}
