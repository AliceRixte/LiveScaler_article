\subsubsection{Inversions}

\begin{figure}[htbp]
  \centering
  \begin{tikzpicture}[baseline= (a).base]

    \node[scale=1] (a) at (0,0){
      \begin{tikzcd}[column sep=0mm, minimum width = 0mm, minimum height=7mm, row sep=0cm]
        \svdots   & \svdots & \hspace{20mm} & \svdots & \svdots \\
        \writechord{G}_{-1}  & 7  & & 7  & \writechord{G}_{-1}  \\
        \writechord{F#}_{-1} & 6  & & 6  & \writechord{F#}_{-1} \\
        \writechord{F}_{-1}  & 5  & & 5  & \writechord{F}_{-1}  \\
        \writechord{E}_{-1}  & 4  & & 4  & \writechord{E}_{-1}  \\
        \writechord{D#}_{-1} & 3  & & 3  & \writechord{D#}_{-1} \\
        \writechord{D}_{-1}  & 2  & & 2  & \writechord{D}_{-1}  \\
        \writechord{C#}_{-1} & 1  & & 1  & \writechord{C#}_{-1} \\
        \writechord{C}_{-1}  & 0  & & 0  & \writechord{C}_{-1}  \\
        \writechord{B}_{-2}  & -1 & & -1 & \writechord{B}_{-2}  \\
        \writechord{A#}_{-2} & -2 & & -2 & \writechord{A#}_{-2} \\
        \writechord{A}_{-2}  & -3 & & -3 & \writechord{A}_{-2}  \\
        \svdots         & \svdots & & \svdots &         \svdots 
        \arrow[from=2-2, to=12-4]
        \arrow[from=4-2, to=10-4, color={rgb,255:red,117;green,117;blue,117}, dashed]
        \arrow[from=5-2, to=9-4]
        \arrow[from=7-2, to=7-4, color={rgb,255:red,117;green,117;blue,117}, dashed]
        \arrow[from=9-2, to=5-4]
        \arrow[from=10-2, to=4-4, color={rgb,255:red,117;green,117;blue,117}, dashed]
        \arrow[from=12-2, to=2-4]
      \end{tikzcd}
    };
  \end{tikzpicture}
  \caption{La transformation $A\langle -1,4\rangle :n\mapsto -n + 4$ correspond au passage à la relative mineure de \writechord{Cma} à \writechord{Ami}}
  \medskip
  \small
  Pour des raisons de lisibilité seules les flèches de la gamme de Do majeur ont été tracées. On notera la passage de l'accord \writechord{Cma} à \writechord{Ami} et de \writechord{Ami} à \writechord{Cma}.
  \label{fig:inversion}
\end{figure}

Lorsque $\mu = -1$, les transformations $A \langle -1,\tau\rangle : n\mapsto -n + \tau$ permettent de passer d'une gamme majeure à une gamme mineure naturelle et réciproquement. La Figure \ref{fig:inversion} illustre la manière dont le transformation $A\langle -1,4 \rangle$ envoie l'accord de Do majeur $C_0,E_0,G_0$ sur l'accord de La mineur $A_{-1},C_0,E_0$ mais aussi envoie l'accord de la mineur $A_{-1},C_0,E_0$ sur l'accord de Do majeur  $C_0,E_0,G_0$. Comme les transformations affines préservent les classes de hauteur, on peut affirmer plus généralement que $A \langle -1,4\rangle$ envoie Do majeur sur La mineur et La mineur sur Do majeur. Le tableau \ref{tab:triadesA-14} explicite l'image des accords de la gamme de Do majeur par $A\langle -1,4 \rangle$.

De manière générale, les transformations affines se comportent moins bien sur les gammes mineures harmoniques. Dans le cas de l'inversion $A\langle -1, 4\rangle$, \writechord{G\sharp} est envoyé sur lui-même et donc l'image de la gamme de La mineur harmonique par $A\langle -1, 4\rangle$ contient les notes \writechord{C}, \writechord{D}, \writechord{E}, \writechord{F}, \writechord{G},\writechord{G\sharp}, \writechord{B}.



\begin{table}[htbp]
  
  \centering % instead of \begin{center}
  \begin{tabular}{ccc}
      \writechord{Cma} & $\mapsto$ & \writechord{Ami}\\
      \writechord{Dmi} & $\mapsto$ & \writechord{Gma}\\
      \writechord{Emi} & $\mapsto$ & \writechord{Fma}\\
      \writechord{Fma} & $\mapsto$ & \writechord{Emi}\\
      \writechord{Gma} & $\mapsto$ & \writechord{Dmi}\\
      \writechord{Ami} & $\mapsto$ & \writechord{Cma}\\
      \writechord{Bo} & $\mapsto$ & \writechord{Bo}
  \end{tabular}
  \caption{Image des accords de la gamme de Do majeur par $A\langle -1, 4 \rangle$\label{tab:triadesA-14}}
\end{table}
