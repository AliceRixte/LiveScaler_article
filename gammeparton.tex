\subsubsection{Transformation vers une gamme par ton}
Lorsque $\mu = 2$ ou $\mu = -2$, les transformations affines envoient n'importe quelle gamme vers une gamme apparentée à une gamme par tons (voir Figure \ref{fig:gammepartons}). Les transformations affines peuvent donc permettre de sortir du cadre de la musique tonale occidentale.

\begin{figure}[htbp]
  \centering
  \begin{tikzpicture}[baseline= (a).base]    

    \node[scale=1] (a) at (0,0){
      \begin{tikzcd}[column sep=0mm, minimum width = 0mm, minimum height=7mm, row sep=0cm]
        \svdots   & \svdots & \hspace{20mm} & \svdots & \svdots \\
        \writechord{G}_{-1}  & 7  & & 7  & \writechord{G}_{-1}  \\
        \writechord{F#}_{-1} & 6  & & 6  & \writechord{F#}_{-1} \\
        \writechord{F}_{-1}  & 5  & & 5  & \writechord{F}_{-1}  \\
        \writechord{E}_{-1}  & 4  & & 4  & \writechord{E}_{-1}  \\
        \writechord{D#}_{-1} & 3  & & 3  & \writechord{D#}_{-1} \\
        \writechord{D}_{-1}  & 2  & & 2  & \writechord{D}_{-1}  \\
        \writechord{C#}_{-1} & 1  & & 1  & \writechord{C#}_{-1} \\
        \writechord{C}_{-1}  & 0  & & 0  & \writechord{C}_{-1}  \\
        \writechord{B}_{-2}  & -1 & & -1 & \writechord{B}_{-2}  \\
        \writechord{A#}_{-2} & -2 & & -2 & \writechord{A#}_{-2} \\
        \writechord{A}_{-2}  & -3 & & -3 & \writechord{A}_{-2}  \\
        \svdots         & \svdots & & \svdots &         \svdots 
        \arrow[from=5-2, to=1-4, color={rgb,255:red,117;green,117;blue,117}, dotted]
        \arrow[from=6-2, to=3-4]
        \arrow[from=7-2, to=5-4]
       \arrow[from=8-2, to=7-4]
       \arrow[from=9-2, to=9-4]
       \arrow[from=10-2, to=11-4]
       \arrow[from=11-2, to=13-4, color={rgb,255:red,117;green,117;blue,117}, dotted]
    \end{tikzcd}
    };
  \end{tikzpicture}   
  \caption{La transformation $A\langle 2,0\rangle :n\mapsto 2n$ permet d'obtenir des gammes apparentées à une gamme par tons.\label{fig:gammepartons}}
\end{figure}

Contrairement aux inversions et aux transpositions, cette transformation n'est pas bijective : l'image de $A\langle 2,0 \rangle$ contient exactement $6$ classes de hauteurs qui correspondent aux $6$ notes d'une des deux gammes par tons. Il est intéressant de noter que l'image d'une gamme majeure ou mineure naturelle par $A\langle 2,\tau\rangle$ contient les $6$ notes de la gamme par tons (voir Table \ref{tab:minparton}). Ce n'est pas le cas pour la gamme mineure harmonique dont l'image par $A\langle 2,\tau\rangle$ ne contient que $5$ classes de hauteur.




\begin{table}[htbp]
  \centering
  \begin{subtable}[t]{0.24\textwidth}
    \centering % instead of \begin{center}
      \begin{tabular}{ccc}
          \writechord{C} & $\mapsto$ & \writechord{C}\\
          \writechord{D} & $\mapsto$ & \writechord{E}\\
          \writechord{E} & $\mapsto$ & \writechord{G\sharp}\\
          \writechord{F} & $\mapsto$ & \writechord{A\sharp}\\
          \writechord{G} & $\mapsto$ & \writechord{D}\\
          \writechord{A} & $\mapsto$ & \writechord{F\sharp}\\
          \writechord{B} & $\mapsto$ & \writechord{A\sharp}
      \end{tabular}
  \end{subtable}%
  \begin{subtable}[t]{0.24\textwidth}
      \centering % instead of \begin{center}
      \begin{tabular}{ccc}
          \writechord{C} & $\mapsto$ & \writechord{C}\\
          \writechord{D} & $\mapsto$ & \writechord{E}\\
          \writechord{E\flat} & $\mapsto$ & \writechord{F\sharp}\\
          \writechord{F} & $\mapsto$ & \writechord{A\sharp}\\
          \writechord{G} & $\mapsto$ & \writechord{D}\\
          \writechord{A\flat} & $\mapsto$ & \writechord{E}\\
          \writechord{B\flat} & $\mapsto$ & \writechord{G\sharp}
      \end{tabular}
    \end{subtable}
    \caption{Image des gammes de Do majeur (à gauche) et Do mineur naturel (à droite) par $A\langle 2, 0 \rangle$\label{tab:minparton}}
\end{table}