\subsection{Théorie transformationnelle}
\begin{comment}
\begin{enumerate}
  \item Les transformations affines sont une généralisation 
  \begin{enumerate}
    \item des Tonnetz
    \item des k-Nets
  \end{enumerate}
  \item ces transformations sont étudiées et utilisées surtout dans un contexte d'analyse et d'aide à la composition (OpenMusic)
\end{enumerate}
\end{comment}

La théorie transformationnelle (pour une introduction généraliste du point de vue mathématique, voir \cite{andreatta2008calcul}, et du point de vue musicologique, lire \cite{andreatta2014introduction}) prend ses racines dans la \emph{Set-Theory}, qui se concentre sur la notion de classe de hauteur, c'est à dire un ensemble de notes identiques à l'octave près (\cite{forte1973structure}). Elle propose une approche plus algébrique que la \emph{Set-Theory}, en se concentrant, entre autre sur la notion de transformation entre ensembles de classes de hauteurs (\cite{lewin1987generalized}).

Les transformations affines sont directement inspirées de la théorie transformationnelle, et plus particulièrement des automorphismes du groupe T/I présentés par \textcite{lewin1990klumpenhouwer}. L'unique différence avec ceux-ci est que les transformations affines ne sont pas nécessairement bijectives et autorisent donc un coefficient modal qui ne soit pas nécessairement premier avec $12$. Les transformations affines sont donc une vision plus terre à terre des automorphismes proposés par Lewin et Klumpenhouwer, tout en proposant quelques transformations supplémentaires sortant du système tonal.

Plusieurs outils permettent OpenMusic  propose une aide à la composition en s'appuyant sur les représentations de la théorie transformationnelle (\cite{andreatta2003formalisation}).   Depuis une dizaine d'année, OpenMusic essaie de concilier l'approche hors du temps de l'aide à la composition avec l'approche temps-réel propre à la performance (\cite{bresson2014reactive}, \cite{bresson2017next}).


