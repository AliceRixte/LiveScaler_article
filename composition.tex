\subsection{Processus de composition}

J'ai composé \emph{Escape} dans le but de le jouer avec LiveScaler. C'est un morceau de trance psychédélique\footnote{La trance psychédélique est souvent  appelée \emph{psytrance}, le lectorat curieux pourra écouter par exemple l'album \emph{The Gathering} (1999) du duo israëlien \emph{Infected Mushroom}.} composé  : 
\begin{itemize}
  \item d'une mélodie minimaliste durant 2 mesures jouée par un synthétiseur  dont le son se rapproche d'un métallophone éthéré
  \item d'une \emph{rolling bass} classique devenue une des signatures de la \emph{psytrance}, et de plusieurs autres basses sur une unique note pédale produisant une ligne de basse riche dans sa texture et son timbre
  \item d'une rythmique séquencée à l'avance indépendante de LiveScaler (pistes audio)
  \item de quelques effets sonores (\emph{risers}, \emph{downshifters} \dots) eux aussi typiques de la \emph{psytrance}, que je déclenche ponctuellement à l'aide du Push.
\end{itemize}

L'objectif était d'une part de partir d'un morceau simple, basé sur un unique accord (ici A mineur) et d'enrichir son harmonie et sa ligne mélodique avec LiveScaler; et d'autre part de proposer un morceau typique d'un genre de musique électronique populaire extrêmement codifié (ici la \emph{psytrance}). Le choix du genre n'est pas anodin :  je souhaitais aussi utiliser un outil de contrôle live expérimental pour un morceu de musique qui, lui, n'a rien d'expérimental. Pour moi, il s'agit plus avec LiveScaler d'explorer de nouvelles modalités live que de nouveaux horizons musicaux.

Pour autant, LiveScaler peut tout à fait intervenir dans le processus de composition. On peut par exemple l'utiliser pour trouver des variations sur une mélodie ou une arpège en expérimentant avec les différents changements de gamme, puis consolider le MIDI une fois qu'on a trouvé une idée satisfaisante. On obtient alors un processus créatif incrémental et prône à la sérendipité partant d'une mélodie ou d'une progression d'accords simple qu'on enrichit ensuite avec LiveScaler.
