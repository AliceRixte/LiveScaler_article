\subsection{Quand appliquer les transformations ? }
Lorsque \emph{Instrument} reçoit la commande d'appliquer une nouvelle transformation de gamme, elle est sensée prendre effet immédiatement et sera appliquée à toutes les notes reçues jusqu'au prochain changement de gamme. Lorsque l'instrument n'est pas en train de jouer, cela ne pose aucune difficulté : il appliquera la transformation au prochain message MIDI qu'il recevra. Il se peut en revanche que l'instrument soit déjà en train de jouer une note. LiveScaler propose $4$ manières de réagir dans une telle situation. 


\begin{enumerate}
  \item \emph{Stop} : toutes les notes en train d'être jouées sont instantanément arrêtées en envoyant un message Note-off pour chaque note en cours. L'instrument reprendra son jeu, en appliquant la nouvelle transformation, au prochain message MIDI qu'il recevra. Cette option est particulièrement adaptée aux instruments dont la durée des notes est courte. 
  \item \emph{Legato} : chaque note en cours est stoppée et instantanément remplacée par son image par la nouvelle transformation. Si l'instrument virtuel est paramétré sur \emph{Legato}, alors les changements de gamme déclencheront des legatos. C'est l'option retenue par \textcite{Livingstone_Muhlberger_Brown_Thompson_2010}.
  \item \emph{ReTrigger} : agit sur le même principe que \emph{Legato} à la différence  qu'un court délai est introduit entre la fin de la note en cours et la note transformée, forçant une nouvelle attaque, même si l'instrument virtuel est en mode legato.
  \item \emph{Wait} : les notes en cours continuent d'être jouées telles quelles. Si elles ne se sont pas arrêtées avant, elles seront stoppée lorsque la prochaine note sera jouée, à partir de laquelle la nouvelle transformation prendra effet.
\end{enumerate}

Le choix entre ces quatre modes se fait de manière locale, deux instances de \emph{Instrument} pourront donc réagir différemment.
