\subsubsection{Coefficient modal}

La Table \ref{tab:classmu} résume les différents types de transformations couvertes par les transformations affines, ainsi que le nombre de classes de hauteur dans l'image de ces transformations \footnote{On montre aisément que le nombre de classe de hauteur dans l'image de $A\langle \mu, \tau\rangle$ est égal à $\frac{12}{\mu\wedge 12}$ où $\wedge$ dénote le pgcd de deux entiers.}. Les transformations affines bijectives - leur image contient $12$ classes de hauteurs - correspondent  aux automorphismes $F\langle u,j \rangle$ du groupe $T/I$ décrits par \cite{lewin1990klumpenhouwer}.


\begin{table}[h]
  \centering
  \rowcolors{2}{gray!25}{white}
  \begin{tabular}{ccc}
    \rowcolor{gray!50}
    $\mu$ & Type de transformation & Classes de hauteur\\
    -1 & Inversions majeur/mineur & 12\\
    0 & Octaves & 1\\
    1 & Transpositions & 12 \\
    -2,2 & Gamme par tons & 6 \\
    -3,3 & Tierces mineures &4 \\
    -4,4 & Tierces majeures & 3\\
    -5,5 & $F\langle 5,\tau \rangle$, $F\langle 7,\tau \rangle$& 12 \\
    -6,6 & Tritons & 2\\
  \end{tabular}
  \caption{Classification des transformations affines en fonction de leur coefficient modal $\mu$\label{tab:classmu} } 
\end{table}