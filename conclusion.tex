Nous avons présenté LiveScaler, qui propose de nouvelles modalités pour la musique électronique live. Live\-Scaler utilise pour l'instant un nombre restreint de transformations MIDI, les transformations affines, qui peuvent être appliquées en live. En particulier, cet outil peut être utilisé dans le contexte de l'EDM, proposant ainsi une alternative ou un complément aux techniques standard de \emph{DJing}.

Plusieurs améliorations techniques pourraient être proposées pour améliorer LiveScaler. En particulier, il serait pertinent de le rendre compatible avec n'importe quel DAW, et pas seulement Ableton Live. Une solution sera de rendre développer un plugin VST ou LV2 pour LiveScaler. De plus, le protocole MIDI étant contraignant, rendre LiveScaler compatible avec un protocole plus flexible tel que OSC \cite{wright2005open}, MPE (Midi Polyphonic Expression), Midi 2.0 ou encore  MP \cite{goudard2017mapping}, qui serait particulièrement adapté à cette application.

Bien que LiveScaler propose déjà la possibilité d'ajouter manuellement des transformations de gamme quelconques, ce mécanisme est laborieux et mérite d'être amélioré. Enfin, nous aimerions pouvoir agir sur d'autres paramètres que la hauteur des notes, en particulier le rythme, le timbre, ou même contrôler simultanément des transformations musicales et vidéos. C'est sur ces axes que seront concentrés nos recherches futures, tout en restant sur le même paradigme de performance live que celui proposé ici.
