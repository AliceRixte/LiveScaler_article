\section{Conclusion}

Tous les outils cités plus haut offrent une grande flexibilité pour appliquer des transformations potentiellement bien plus sophistiquées que les transformatiions affines, et sont a priori tous capables de le faire en live. Une spécificité de LiveScaler est l'idée de d'utiliser ces transformations comme un instrument de musique, et de les rendre directement accessibles aux musiciens même s'ils n'ont aucune affinité avec la programmation.

Utiliser la flexibilité et la rapidité de bell (\cite{agostini2020programmer}). Utiliser un meilleur protocole que MIDI (OSC,MP \cite{goudard2017mapping})
\subsection{Perpectives}
\begin{enumerate}
  \item Associer les changements de gammes à un contrôle d'image (VJing)
  \item Proposer un macro contrôle pour le rythme
  \item Obtenir le retour utilisateurs de musiciens et performeurs
\end{enumerate}