\subsection{Transformations affines}

Les transformations de gamme qui vont nous intéresser ici sont les \emph{transformation affines}, c'est-à-dire les fonctions de la forme $A\langle\mu,\tau\rangle : n \mapsto \mu n + \tau$ avec $\mu$ le \emph{coefficient modal} de la transformation affine et $\tau$ le \emph{coefficient de transposition}. Le coefficient multiplicateur $\mu$ détermine le mode dans lequel on se trouvera après avoir appliqué la transformation. 

Les transformations affines ont la propriété importante de préserver les classes de hauteur \footnote{En effet, pour toute base $\beta\in \mathbb{N}^*$, $\forall n_1,n_2 \in \mathbb{Z}, n_1 \equiv n_2 \mod \beta \implies \mu n_1 + \tau \equiv \mu n_2 + \tau \mod \beta$. Avec $\beta=12$, on obtient le résultat pour les classes de hauteurs dodécaphoniques. }  (au sens de Forte \cite{forte1973structure}) c'est-à-dire que si deux notes sont identiques à l'octave près, alors elles le seront toujours une fois la transformation affine appliquée. 

Nous allons à présent étudier plusieurs exemples afin de donner au lecteur un aperçu de leur expressivité. Dans l'ensemble de ces exemples, on se placera dans le tempérament égal $T\langle $\writechord{C}$_5,12\rangle$ : la note \writechord{C}$_5$ correspondra ainsi à l'entier $0$, \writechord{C#}$_5$ à $1$, \writechord{C}$_6$ à $12$, \writechord{B}$_4$ à $-1$, \writechord{C}$_4$ à $-12$, etc. Nous nous concentrerons ici sur les cas $\mu = 1$, $\mu = -1$ et $\mu = 2$.
