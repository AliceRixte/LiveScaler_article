\subsubsection{Transpositions}

\begin{figure}[htbp]
  \centering
  \begin{tikzpicture}[baseline= (a).base]

    \node[scale=1] (a) at (0,0){
      \begin{tikzcd}[column sep=0mm, minimum width = 7mm, minimum height=7mm, row sep=0cm]
        \svdots   & \svdots & \hspace{20mm} & \svdots & \svdots \\
        \writechord{E}_{-1}  & 4  & & 4  & \writechord{E}_{-1}  \\
        \writechord{D#}_{-1} & 3  & & 3  & \writechord{D#}_{-1} \\
        \writechord{D}_{-1}  & 2  & & 2  & \writechord{D}_{-1}  \\
        \writechord{C#}_{-1} & 1  & & 1  & \writechord{C#}_{-1} \\
        \writechord{C}_{-1}  & 0  & & 0  & \writechord{C}_{-1}  \\
        \writechord{B}_{-2}  & -1 & & -1 & \writechord{B}_{-2}  \\
        \writechord{A#}_{-2} & -2 & & -2 & \writechord{A#}_{-2} \\
        \writechord{A}_{-2}  & -3 & & -3 & \writechord{A}_{-2}  \\
        \svdots         & \svdots & & \svdots &         \svdots 
        \arrow[from=3-2, to=1-4, dotted]
        \arrow[from=4-2, to=2-4]
        \arrow[from=5-2, to=3-4]
        \arrow[from=6-2, to=4-4]
        \arrow[from=7-2, to=5-4]
        \arrow[from=8-2, to=6-4]
        \arrow[from=9-2, to=7-4]
        \arrow[from=10-2, to=8-4, dotted]
      \end{tikzcd}
    };
  \end{tikzpicture}
  \caption{La transformation $A \langle 1,2 \rangle : n \mapsto n + 2$ correspond à la  transposition d'un ton vers l'aigu}
  \label{fig:transp}
\end{figure}
Lorsque $\mu = 1$, les transformations affines $A\langle 1,\tau \rangle : n \mapsto n + \tau$ permettent de représenter toutes les transpositions possibles (voir Figure \ref{fig:transp}).

